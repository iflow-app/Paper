\chapter[Proposta]{Proposta}

Este capítulo visa detalhar a proposta deste Trabalho de Conclusão de Curso, que se baseia na construção de uma ferramenta orientada ao gerenciamento e suporte ao processo da Engenharia de Requisitos. Portanto, a primeira seção apresenta a contextualização da proposta e a ferramenta \textit{iTrace}, com a arquitetura de \textit{software} e as especificidades dos módulos do sistema. Posteriormente, será apresentada a identidade visual da ferramenta junto ao protótipo de alta fidelidade. Por fim, apresentam-se as páginas e as dinâmicas que tornam viável a semi-automatização dos processos da Engenharia de Requisitos.

\section{Contextualização}
A principal justificativa para o desenvolvimento da ferramenta proposta neste trabalho diz respeito a buscar minimizar e evidenciar os problemas associados aos processos da Engenharia de Requisitos.

Reproduzindo um papel essencial no desenvolvimento de produtos de \textit{software}, a Engenharia de Requisitos fornece uma maneira de entender e descrever problemas que precisam de uma solução de \textit{software}. Além disso, é um ramo que está inteiramente ligado no ciclo de desenvolvimento do \textit{software} \cite{elliott2012software}. Dessa forma, por se tratar de uma área negligenciada por muitos, foi considerada a construção de uma ferramenta que ofereça apoio tecnológico, que viabilize e semi-automatize as etapas da Engenharia de Requisitos, com o propósito de facilitar e melhorar estes processos.

A seguir, tem-se uma descrição mais aprofundada da ferramenta que está sendo elaborada para o Trabalho de Conclusão de Curso.

\section{Sobre a Ferramenta iTrace}
A ferramenta \textit{iTrace} tem o objetivo de viabilizar e semi-automatizar o processo da Engenharia de Requisitos, de modo a embasar a construção de um produto de \textit{software} e evitar que retrabalhos e desperdícios de tempo ocorram devido ao mau planejamento. Neste sentido, algumas automatizações serão realizadas, de modo a estabelecer e se apoiar no conceito de um \textit{Minimum Viable Product} (MVP). Além disso, a ferramenta dá ênfase nas etapas da Engenharia de Requisitos, de forma que, uma vez concluída uma etapa, a próxima será desbloqueada. Dessa forma, a ferramenta será composta por algumas etapas, dentre elas: a) pré-rastreabilidade; b) elicitação; c) modelagem; d) \textit{house of quality}; e) análise; e f) pós-rastreabilidade. 

\subsection{Pré-rastreabilidade}

Nesta primeira etapa, o foco da ferramenta será em aplicar os conceitos propostos dentre deste contexto (seção \ref{sec:pre-rastreabilidade}) e iniciar o processo da Engenharia de Requisitos, para descrever as primeiras definições do produto de \textit{software} em questão, para iniciar o levantamento dos requisitos.

Com esse objetivo, os artefatos usados para esta etapa são o \textit{5W2H} (seção \ref{sec:5w2h}, que sintetiza a ideia inicial do produto, e o \textit{Rich Picture} (seção \ref{sec:rich_picture}, que transcreve visualmente os agentes e fluxo de informações envolvidos no processo.

\subsection{Elicitação}

Construído e definido a visão geral do produto de \textit{software}, a ferramenta parte para a segunda etapa do processo, que levanta do maior número possível de features. A intenção é que se colete diferentes pontos de vistas em diferentes contextos com opiniões distintas. Com esses diferentes aspectos espera-se ter features mais ricas;
Várias artefatos serão dados ao usuário para o construção dessa etapa, contudo não será obrigatório que o mesmo faça todos para seguir ao próximo nível;
Pode-se ter uma sugestão ou valor mínimo de features para que o usuário passe para a próxima etapa;
A proposta é que o usuário tenha a possibilidade de adicionar diferentes imagens, vídeos, áudios e documentos de texto e, assim, possa extrair novas features;
Para embasar o usuário, pequenas definições serão colocadas ao lado de cada artefato de forma a guiar a geração daquele artefato;
Como possibilidade, pode-se ter documentos externos que tenham uma boa definição do que é o artefato caso o usuário queira aprender;


\section{Arquitetura de Software}
Conforme é observado na figura \ref{fig:arquitetura}, a arquitetura de \textit{software} proposta inclui a integração de 2 módulos: o \textit{front-end} e o \textit{back-end}.

\subsection{\textit{Front-end}}
O \textit{Front-end} será responsável por realizar as interações com o usuário e com o \textit{back-end}. Além disso, é responsável pela \textit{interface} da ferramenta utilizando recursos \textit{HTML} e \textit{CSS} com o \textit{framework} \textit{React}. Essas tecnologias foram escolhidas devido ao grande suporte que elas oferecem para a construção de \textit{interfaces} de usuário interativas, e são descritas com mais detalhes no capítulo \ref{chap:referencial_tecnologico}.

\subsection{\textit{Back-end}}


\section{Identidade Visual}

\subsection{Paleta de Cores}

\subsection{Tipografia}

\subsection{Logotipo}