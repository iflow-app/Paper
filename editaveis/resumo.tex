\begin{resumo}
    A Engenharia de Requisitos é um processo essencial no desenvolvimento de um produto de \textit{software}, possuindo diversas etapas a serem seguidas, e sendo fundamental para a elaboração do produto. Dada a relevância da área, e as diferentes vertentes de atuação demandadas para cumprimento de suas atividades, os Engenheiros de Requisitos e profissionais afins revelam dificuldades, portanto, necessidades de recursos que os auxiliem. Visando corroborar nesse contexto, esse trabalho propõe um apoio guiado, chamado \textit{iFlow}, orientado à literatura especializada, que procura automatizar parte desse processo. O \textit{iFlow} é um \textit{software} web desenvolvido em \textit{NodeJS} e \textit{ReactJS}, direcionado a um público técnico, e centrado nas principais atividades da Engenharia de Requisitos, sendo elas: Pré-rastreabilidade, Elicitação, Modelagem, Análise (foco em Verificação e Priorização), e Pós-Rastreabilidade. A automatização no \textit{iFlow} orienta-se por um processo de planejamento sistemático, conhecido como \textit{House of Quality}, cuja implementação é viabilizada com uso de Grafos, e Algoritmos de Menor Caminho. O iFlow é capaz de gerar uma versão preliminar de um Produto Mínimo Viável.
    \vspace{\onelineskip}
    \noindent
    \\
    \textbf{Palavras-chave}: Engenharia de Requisitos. Elicitação. Modelagem. Análise. Rastreabilidade. \textit{House of Quality}. Grafos. Algoritmos de Menor Caminho.
\end{resumo}
