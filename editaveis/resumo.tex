\begin{resumo}
%A Diabetes Mellitus é um problema de saúde que atinge 6,9\% da população brasileira, equivalente a 13 milhões de pessoas, destes 6,5 milhões não sabem que possuem a doença metabólica. Neste sentido, é descrita uma proposta de desenvolvimento de um novo método, não invasivo, para o diagnóstico da diabetes. Assim, sabe-se que uma das alterações metabólicas causadas pela diabetes é a produção e consequente presença de vapores de acetona no hálito com concentração maior que ca. 2 ppm. O presente estudo tem como objetivo, apresentar uma proposta de modelo matemático, utilizando a técnica Bond Graph, do processo de emissão de gás da engenharia do Sistema Respiratório, obtendo a representação matemática que descreva o hálito cetônico, dando subsídios para a diferenciação de um possível perfil diabético de um perfil saudável. Desta forma, a metodologia empregada para a determinação do modelo matemático foi dividida em 5 etapas: i) detalhamento do sistema real; ii) especificação do sistema análogo; iii) definição das hipóteses simplificadoras; iv) aplicação da técnica Bond Graph; e v) obtenção da representação em espaços de estados do sistema com relação a entrada: difusão da cetona do plasma sanguíneo nos pulmões, e saída: presença de acetona no hálito. Com objetivo de analisar o sistema, foram empregadas técnicas de controle para avaliar o comportamento dinâmico do sistema utilizando análises, no domínio da frequência e do tempo para verificar a estabilidade do sistema. Diante da modelagem matemática do sistema obtida por meio do modelo análogo, permitiu-se uma maior compreensão das variáveis físicas envolvidas na emissão do hálito cetônico, identificando os parâmetros de estabilidade, tornando viável o controle do sistema por meio da representação em espaço de estados. Perante tais resultados, será possível, como próximo passo, o desenvolvimento de tecnologias assistivas com controle aplicado na construção de sensores de gás.
 \vspace{\onelineskip}
    
 \noindent
 \textbf{Palavras-chaves}: Modelagem Matemática, Diabetes, Bond Graph, Hálito Cetônico.
\end{resumo}
