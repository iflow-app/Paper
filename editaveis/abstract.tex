\begin{resumo}[Abstract]
 \begin{otherlanguage*}{english}
    Requirements Engineering is an essential part in software product development. It has many steps to be followed and is fundamental in product elaboration. Given the importance of this field, and the different action strands demanded to fulfil the activities, Requirement Engineers and correlated professionals reveal some difficulties and, therefore, require resources for assistance. In this perspective, this paper proposes a guided support, called \textit{iFlow}, based on specialized literature that aims to automatize part of this process. The \textit{iFlow} is a software developed in NodeJS and ReactJS, directed to a specialized audience, and centered in the main activities of Requirements Engineering: Pre-traceability, Elicitation, Modeling, Analysis (focused in Verification and Prioritization) and Post-traceability. The \textit{iFlow} automatization is guided by a systematic planned process, known as House of Quality, that will be implemented using Graphs, and Single Shortest Path algorithms. The idea is to generate a preliminary version of the Minimum Viable Product.
   \vspace{\onelineskip}
   \noindent
   \\
   \textbf{Key-words}: Requirements Engineering, Elicitation, Modeling, Analyze, Traceability, House of Quality, Graphs, Single Shortest Path Algorithm.
 \end{otherlanguage*}
\end{resumo}
