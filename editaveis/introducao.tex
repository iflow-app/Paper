\chapter[Introdução]{Introdução}

\section{Contextualização}

A Engenharia de Requisitos é um ramo da Engenharia de \textit{Software} que se preocupa com os objetivos do mundo real, funções e restrições dos sistemas de \textit{software}, fazendo parte do ciclo de desenvolvimento do \textit{software}, e sendo crucial para o sucesso geral e precisão do sistema desenvolvido. Esse ramo também se preocupa com a relação desses fatores com caracterizações exatas do comportamento do \textit{software} e com a sua evolução ao longo do tempo \cite{elliott2012software}.

Desempenhando um papel significativo no desenvolvimento de aplicativos de \textit{software}, a Engenharia de Requisitos fornece uma maneira de entender e descrever os problemas que requerem uma solução de \textit{software}. Isso é crucial para compreender as necessidades dos usuários e especificar tais necessidades como requisitos de \textit{software} \cite{elliott2012software}. É uma área de suma importância, pois é necessária para melhorar as possibilidades de sucesso de um projeto.

A elicitação, a modelagem e a análise (verificação e validação) são processos essenciais dentro da Engenharia de Requisitos, os quais ajudam a cumprir o propósito de um \textit{software}. Além disso, a Engenharia de Requisitos documenta o sistema, identifica todas as partes interessadas e suas preocupações e apresenta essas informações de uma forma que possam ser analisadas e, eventualmente, implementadas \cite{elliott2012software}.

\section{Problemática}
    
Um dos grandes casos estudados na área de Engenharia de Requisitos é o sistema de \textit{software} desenvolvido para a automatização do Serviço de Ambulância de Londres (SAL). Este serviço tinha uma taxa de 2000 a 2500 ligações diárias, com uma frota de 750 veículos e área coberta de aproximadamente 1600 km². Foi operado manualmente até o ano de 1987, quando o sistema foi posto em funcionamento \cite{LodonFiasco}.

O sistema ficou em operação por menos de 10 dias, e se mostrou totalmente ineficiente para as demandas do serviço que propunha automatizar. Foi causa da morte de, aproximadamente, 30 pessoas; gerou um prejuízo em torno de £1.5 milhões; casou a resignação do chefe executivo do SAL; instaurou um inquérito na equipe que desenvolveu o produto, e deixou em voga a responsabilização ética para profissionais de TI, como para outras áreas, tais como direito e medicina \cite{LodonFiasco}.

Os principais pontos de erros levantados no trabalho de \cite{LodonFiasco}, levando em consideração os problemas de elicitação do projeto, foram:

\begin{itemize}
    \item Prolemas na definição do cronograma de execução do projeto, dado que foi definido um prazo irreal para o desenvolvimento do sistema, treinamento de pessoal e testes do \texit{software};
    
    \item O comitê responsável teve um papel mínimo no desenvolvimento do sistema, uma vez que, metade das pessoas ficaram responsáveis por grande parte do trabalho, e não levaram em conta a visão dos funcionários que trabalhavam dentro da ambulância. Isso levou a um cenário de forte introspecção, com poucos pontos de vista do sistema, criando algo que não atendia as necessidades e também não levava em consideração o contexto que estava inserido;
    
    \item Não levou em consideração situações de erro dentro da captação de dados da situação e localização das ambulâncias ou mesmo de recursos de sistema necessários para que o sistema responsável para enviar ambulâncias em locais de incidentes pudesse funcionar corretamente. Em períodos com sobrecarga de chamados, o sistema colapsava e levava bastante tempo para executar;
    
    \item Realizaram o treinamento inadequado, de forma que, os funcionários tiveram que se adaptar e compreender um sistema que ainda não estava totalmente pronto. Vale ressaltar que o \texit{software} teve uma série de modificações antes de ser finalizado, e
    
    \item O sistema foi testado, apenas, dentro do contexto de cada módulo em separado. Contudo, não foi testado a integração dos módulos e realizados os \textit{testes de estresse}\footnote{Definição}, que são vitais dentro do contexto de uma aplicação tão crítica.
\end{itemize}

A partir do exposto, nota-se o quão importante é uma Engenharia de Requisitos feita de forma adequada, assim como o quanto está diretamente associada ao sucesso do sistema desenvolvido. Impacta diretamente nos custos e no resultado final do produto. Um estudo realizado por \textit{Donald Firesmith} mostra que as empresas americanas têm um prejuízo acima de 30 bilhões de dólares por ano, acarretando em projetos fracassados \cite{king2008cost}.

\section{Questão de Pesquisa}

Para guiar as pequisas e o desenvolvimento deste estudo, foram definidas as seguintes indagações:

\begin{enumerate}
    \item Como pode ser automatizado a \textit{hiperlinkagem} para que os rastros das ideias não se percam no processo da engenharia de requisitos?;
    \item Como prover uma interface intuitiva que proporcione dados para realizar a \textit{hiperlinkagem} e gerar o MVP?, e
    \item Como proporcionar uma experiência mais simples e intuitiva para que os desenvolvedores possam fazer o processo da Engenharia de Requisitos de forma mais fácil?.
\end{enumerate}

\section{Objetivos}

\subsection{Objetivo Geral}

Desenvolver uma ferramenta que viabilize a automatização e auxilie todo o processo de Engenharia de Requisitos, de forma a embasar a construção de uma aplicação e evitar que retrabalhos e desperdícios de tempo ocorram devido ao mal planejamento. A ferramenta funcionará de forma a dar ênfase nas etapas da Engenharia de Requisitos, de forma que, uma vez concluída uma etapa, a próxima é desbloqueada.

\subsection{Objetivos Específicos}

\begin{enumerate}
    \item Desenvolver uma aplicação que automatize a \textit{hiperlinkagem} entre os artefatos de requisitos;
    \item Desenvolver uma aplicação que viabilize a geração do MVP em cima dos artefatos gerados;
    \item Criar uma interface que guie o usuário no processo de criação dos artefatos de requisitos mapeados, e
    \item Elaborar uma interface que seja intuitiva e proporcione uma boa experiência de usuário para facilitar o processo da Engenharia de Requisitos.
\end{enumerate}

\section{Organização da Monografia}

O trabalho foi dividido de acordo com o seguinte formato:

\begin{itemize}
    \item Capítulo 2 - Embasamento Teórico: apresenta o que existe na literatura no que permeia e da suporte ao tema do trabalho;
    \item Capitulo 3 - Referencial Tecnológico: caracteriza as tecnologias que serão usadas para o desenvolvimento desta monografia;
    \item Capítulo 4 - Metodologia: Define como será realizada a condução do trabalho, desde a pesquisa, até a construção do \textit{framework} e análise dos resultados;
    \item Capítulo 5 - Proposta: Contém a proposta de funcionamento do \textit{framework}, definições importantes de arquitetura e interface e o \textit{backlog}\footnote{O termo é explicado na seção \ref{sec:backlog} deste trabalho} do produto, e
    \item Capítulo 6 - Resultado Parciais: Apresenta os principais resultados obtidos nesse primeiro momento do desenvolvimento da monografia.
\end{itemize}