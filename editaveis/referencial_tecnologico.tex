\chapter[Referencial Tecnológico]{Referencial Tecnológico}

Este capítulo apresenta as tecnologias que dão suporte para a construção do presente trabalho. As escolhas se deram pela afinidade da dupla e conhecimento prévio, de necessidade e de uso da comunidade, das mesmas para minimização de riscos no desenvolvimento do \textit{framework}.

\section{Tecnologias de Apoio à Aplicação}

\subsection{JavaScript}
O \textit{JavaScript} (JS) é uma linguagem dinâmica, interpretada e multiparadigma, que suporta estilos orientados a objetos, imperativos e declarativos \cite{javascript}. Para o desenvolvimento da aplicação, a linguagem será utilizada a partir do emprego do \textit{NodeJS} e do \textit{ReactJS} como ferramentas de apoio.

\subsection{ReactJS}
O \textit{ReactJS} é uma biblioteca \textit{JavaScript} com o objetivo de criar interfaces de usuário \cite{reactjs}. O \textit{ReactJS} foi escolhido por conta do seu bom desempenho com virtual DOM (\textit{Document Object Model}) e pela facilidade de criação de componentes que ele fornece. Além disso, ele utiliza visualizações declarativas, que torna o código mais previsível e fácil de depurar.

\subsection{NodeJS}
O \textit{NodeJS} é uma plataforma projetada para o desenvolvimento de aplicações escaláveis de rede e é um ambiente de execução de código \textit{JavaScript} \cite{nodejs}. O NodeJS será utilizado para a implementação da \textit{API} que fará a comunicação entre as camadas da aplicação.

\subsection{GitHub}
O \textit{GitHub} é uma plataforma que auxilia todo o ciclo de desenvolvimento de uma aplicação, hospedando todo o código fonte \cite{github}. A plataforma será utilizada para armazenar e centralizar todo o código fonte da aplicação.

\subsection{Git}
O \textit{Git} é uma ferramenta de controle de versão gratuita e de código aberto\cite{git}. Será utilizado para realizar o versionamento de todo o código fonte da aplicação.

\subsection{Docker}
O \textit{Docker} é uma plataforma para a criação, execução e publicação de contêineres \cite{docker}. A ferramenta será utilizada para a gerência de configuração da aplicação, com o objetivo de facilitar e otimizar as atividades de desenvolvimento e de implantação com a utilização dos contêineres.

\subsection{Docker Compose}
O \textit{Docker Compose} é uma ferramenta para definir e executar aplicativos \textit{Docker} de vários contêineres. Com o Compose, você usa um arquivo \textit{YAML} para configurar os serviços da sua aplicação \cite{docker-compose}. O \textit{Docker Compose} será utilizado para a orquestração dos contêineres dos serviços utilizados pela aplicação.

\subsection{Trello}
O Trello é uma ferramenta de gerenciamento de projetos que utiliza um esquema de listas, cartões e quadros para organizar atividades dentro de um projeto \cite{trello}. A ferramenta será utilizada para a criação de um quadro \textit{Kanban}, para que haja o acompanhamento das atividades à serem entregues.

\section{Tecnologias de Apoio à Pesquisa e à Comunicação}
\subsection{Drive}
O \textit{Drive} é uma plataforma criada pelo \textit{Google} e consiste em armazenar, compartilhar e colaborar arquivos e pastas \cite{drive}. O \textit{Google Drive} será utilizado para a elaboração de planilhas e documentos colaborativos referentes ao projeto.

\subsection{LaTeX LaTeX3}
O \textit{LaTeX} é um sistema de preparação de documentos para composição tipográfica de alta qualidade \cite{latex}. O \textit{LaTeX} está sendo utilizado pelo fato de auxiliar na formatação do documento.

\subsection{Overleaf Community Edition}
O \textit{Overleaf} é uma ferramenta \textit{online} de escrita em \textit{LaTeX} e publicação colaborativa para a elaboração de documentos técnicos e científicos \cite{overleaf}. O \textit{Overleaf} será utilizado para a produção e edição do texto científico.

\subsection{Telegram}
O \textit{Telegram} é um aplicativo, disponível nas versões \textit{mobile} e \textit{web}, para o compartilhamento de mensagens. Além de ter suporte ao envio de mensagens de texto, de voz, de mídia e de documentos de todo o tipo, os usuários podem fazer chamadas de vídeo e ligações \cite{telegram}. O aplicativo será utilizado para o compartilhamento de arquivos, links, marcar reuniões, alertas e discussões imediatas.

\subsection{Discord}
O \textit{Discord} é uma plataforma de comunicação que permite a efetuação de chamadas e a realização de videochamadas \cite{discord}. A plataforma será utilizada para a realização de reuniões remotas entre os desenvolvedores do projeto.

\subsection{Notion}
O \textit{Notion} é uma plataforma que é utilizada para criar sistemas de gerenciamento de conhecimento, tomada de notas, gerenciamento de dados, anotações, gerenciamento de projetos, entre outros \cite{notion}. A ferramenta será utilizada para que haja uma organização e separação das atividades, além de salvar todos os artigos que serão utilizados como base para o desenvolvimento do projeto.