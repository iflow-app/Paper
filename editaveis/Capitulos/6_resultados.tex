\chapter[Considerações Finais]{Considerações Finais}

\label{chap:consideracoes_finais}

Este capítulo apresenta os resultados parciais das atividades relacionadas ao Trabalho de Conclusão de Curso. Dessa forma, em um primeiro momento, é apresentado o andamento da primeira etapa do trabalho (seção \ref{sec:andamento_do_trabalho}), desde a Introdução até a Proposta e, em seguida, a proposta da evolução futura, procurando conferir o estado atual, quanto ao desenvolvimento da ferramenta \textit{iFlow} (seção \ref{sec:evolucao_futura}).

\section{Andamento do Trabalho}

\label{sec:andamento_do_trabalho}

Na primeira etapa deste trabalho, focou-se no desenvolvimento das atividades que fundamentam e sistematizam a proposta de projeto em relação à criação de uma ferramenta que confira apoio e semiautomatize as etapas relacionadas ao processo de Engenharia de Requisitos. Nesse sentido, a Tabela \ref{tab:atividades_tcc_1} mostra o andamento das atividades da fase inicial do trabalho, desde a definição do tema até a formulação da proposta, sendo todas tarefas já concluídas.

\begin{table}[H]
    \centering
    \caption{Andamento das atividades do TCC1}
    \scalebox{0.9}{%
    \begin{tabular}{l*{2}{c}r}
        \hline
        Atividade & Andamento \\
        \hline
        Definir Tema & Concluída \\
        Formular Proposta & Concluída \\
        Contextualizar Problema & Concluída \\
        Realizar Levantamento Bibliográfico & Concluída \\
        Definir Referencial Tecnológico & Concluída \\
        Definir Metodologia & Concluída \\
        Refinar Proposta & Concluída \\
        Desenvolver o Protótipo da Ferramenta & Concluída \\
        Revisar TCC1 & Concluída \\
        Definir \textit{Backlog} & Concluída \\
        Apresentar TCC1 & A fazer \\
        \hline
    \end{tabular}}
    \legend{Fonte: Autores, 2022.}
    \label{tab:atividades_tcc_1}
\end{table}

Na fase final deste trabalho, o foco será no desenvolvimento da ferramenta no que diz respeito a realizar os processos da Engenharia de Requisitos de forma semi-automatizada. A Tabela \ref{tab:atividades_tcc_2} apresenta o andamento das atividades relacionadas ao TCC 2.

\begin{table}[H]
    \centering
    \caption{Andamento das atividades do TCC2}
    \scalebox{0.8}{%
    \begin{tabular}{l*{2}{c}r}
        \hline
        Atividade & Andamento \\
        \hline
        Corrigir Apontamentos da Banca & A fazer \\
        Desenvolvimento da Ferramenta & A fazer \\
        Análise dos Resultados & A fazer \\
        Revisar TCC2 & A fazer \\
        Apresentar TCC2 & A fazer \\
        \hline
    \end{tabular}}
    \legend{Fonte: Autores, 2022.}
    \label{tab:atividades_tcc_2}
\end{table}

\section{Evolução Futura}

\label{sec:evolucao_futura}

Considerando que a ferramenta proposta, em seu \textit{status} atual, consta planejada, em termos conceituais (Capítulo \ref{chap:embasamento_teorico}); tecnológicos (Capítulo \ref{chap:referencial_tecnologico}); metodológicos (Capítulo \ref{chap:metodologia}), e arquiteturais \& visuais (Capítulo \ref{chap:proposta}), para a próxima etapa do trabalho, pretende-se desenvolver a ferramenta, orientando-se por esse planejamento, pretende-se aplicar o método de Pesquisa-Ação, para analisar os resultados obtidos com o desenvolvimento da ferramenta, e realizar as devidas correções.