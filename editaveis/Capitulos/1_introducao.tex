\chapter[Introdução]{Introdução}

Neste capítulo, pretende-se contextualizar a pesquisa a ser realizada neste trabalho, sendo a mesma focada na Engenharia de Requisitos e seus processos (seção \ref{ref:contextualizacao}). Dentro deste contexto, é apresentado, primeiramente, a problemática no domínio da Engenharia de Requisitos (seção \ref{ref:problematica}), seguido pelas questões de pesquisa que norteiam a revisão da literatura, bem como pretendem ser respondidas ao final do trabalho (seção \ref{ref:questao_pesquisa}), continuando com os principais objetivos a serem atingidos ao longo da pesquisa (\ref{ref:objetivos}) e, por fim, a organização da monografia (seção \ref{ref:organizacao}).

\label{chap:intro}

\section{Contextualização}

\label{ref:contextualizacao}

A Engenharia de Requisitos é um ramo da Engenharia de \textit{Software} que se preocupa com os objetivos reais, funções e restrições dos sistemas de \textit{software}, fazendo parte do ciclo de desenvolvimento do \textit{software}, e sendo fundamental para o sucesso geral do sistema desenvolvido. Esse ramo também se preocupa com a relação desses fatores com caracterizações exatas do comportamento do \textit{software} e com a sua evolução temporal \cite{elliott2012software}.

Desempenhando um papel significativo no desenvolvimento de aplicativos de \textit{software}, a Engenharia de Requisitos fornece uma maneira de entender e descrever os problemas que requerem uma solução de \textit{software}. Isso é relevante para compreender as necessidades dos usuários e especificar tais necessidades como requisitos de \textit{software} \cite{elliott2012software}. Trata-se de uma área de suma importância, pois é necessária para melhorar as possibilidades de sucesso de um projeto.

A elicitação, a modelagem e a análise (verificação e validação) são processos essenciais na Engenharia de Requisitos, os quais ajudam a cumprir o propósito de um \textit{software}. Além disso, a Engenharia de Requisitos documenta o sistema, identifica todas as partes interessadas e suas preocupações, e apresenta essas informações de uma forma que possam ser analisadas e, eventualmente, implementadas \cite{elliott2012software}.

\section{Problemática}
\label{ref:problematica}
    
Um dos grandes casos estudados na área de Engenharia de Requisitos é o sistema de \textit{software} desenvolvido para a automatização do Serviço de Ambulância de Londres (SAL). Este serviço tinha uma taxa de 2000 a 2500 ligações diárias, com uma frota de 750 veículos e área coberta de cerca de 1600 km². Foi operado manualmente até o ano de 1987, quando o sistema foi posto em funcionamento \cite{LodonFiasco}.

O sistema ficou em operação por menos de 10 dias, e se mostrou totalmente ineficiente para as demandas do serviço que propunha automatizar. Foi causa da morte de, cerca de 30 pessoas; gerou um prejuízo em torno de £1.5 milhões; casou a resignação do chefe executivo do SAL; instaurou um inquérito na equipe que desenvolveu o produto, e deixou em voga a responsabilização ética para profissionais de TI, bem como para outras áreas, dentre elas: direito e medicina \cite{LodonFiasco}.

Os principais pontos de erros levantados no trabalho de \cite{LodonFiasco}, considerando os problemas de elicitação do projeto, foram:

\begin{itemize}
    \item Prolemas na definição do cronograma de execução do projeto, dado que foi definido um prazo irreal para o desenvolvimento do sistema, treinamento de pessoal e testes do \textit{software};
    
    \item O comitê responsável teve um papel mínimo no desenvolvimento do sistema, visto que, metade das pessoas ficaram responsáveis por grande parte do trabalho, e não consideraram a visão dos funcionários que trabalhavam na ambulância. Isso levou a um cenário de forte introspecção, com poucos pontos de vista do sistema, criando algo que não atendia as necessidades e também não considerava o contexto inserido;
    
    \item Não considerou situações de erro na captação de dados da localização das ambulâncias ou mesmo de recursos necessários para que o sistema responsável por enviar ambulâncias em locais de incidentes pudesse funcionar corretamente. Em períodos com sobrecarga de chamados, o sistema colapsava e levava bastante tempo para executar;
    
    \item Realizaram o treinamento inadequado, de forma que, os funcionários tiveram que se adaptar e compreender um sistema que ainda não estava totalmente pronto. Vale ressaltar que o \textit{software} teve uma série de modificações antes de ser finalizado, e
    
    \item O sistema foi testado, apenas, no contexto de cada módulo, ou seja, em separado. Contudo, não foi testada a integração dos módulos, assim como não realizados os \textit{testes de estresse}, adequados no contexto de uma aplicação tão crítica.
\end{itemize}

A partir do exposto, nota-se o quão importante é uma Engenharia de Requisitos elaborada adequadamente, assim como o quanto está diretamente associada ao sucesso do sistema desenvolvido. Impacta diretamente nos custos e no resultado do produto. Corroborando com esse ponto, há um estudo realizado por \textit{Donald Firesmith}, mostrando que as empresas americanas têm um prejuízo acima de 30 bilhões de dólares por ano, acarretando projetos fracassados \cite{king2008cost}.

\section{Questão de Pesquisa}

\label{ref:questao_pesquisa}

Para guiar as pequisas e o desenvolvimento deste estudo, foram definidas as seguintes indagações:

\begin{enumerate}
    \item Como pode ser automatizada a \textit{hiperlinkagem} para que os rastros das ideias não se percam no processo da Engenharia de Requisitos?;
    \item Como prover uma \textit{interface} suficientemente satisfatória visando gerar uma versão preliminar do \textit{Minimum Viable Product} (MVP)?, e
    \item Como proporcionar uma abordagem mais simples e minimalista para que os desenvolvedores possam realizar mais facilmente o processo da Engenharia de Requisitos?.
\end{enumerate}

\section{Objetivos}

\label{ref:objetivos}

\subsection{Objetivo Geral}

Desenvolver uma ferramenta que viabilize automatizações e auxilie no processo de Engenharia de Requisitos, para embasar a construção de uma aplicação e mitigar retrabalhos e desperdícios de tempo. A ferramenta funcionará conferindo ênfase nas etapas da Engenharia de Requisitos, de forma que, uma vez concluída uma etapa, a próxima é desbloqueada, permitindo que o usuário volte em etapas anteriores para que haja um refinamento da mesma.

\subsection{Objetivos Específicos}

\begin{enumerate}
    \item \label{oe_hiperlinkagem} Desenvolver uma ferramenta que automatize a \textit{hiperlinkagem} entre os artefatos de requisitos;
    \item \label{oe_mvp} Evoluir a ferramenta para que a mesma viabilize a geração do MVP (versão preliminar) com base nos artefatos gerados;
    \item \label{oe_guiar_usuario} Prover uma \textit{interface} para a ferramenta que guie o usuário no processo de criação de alguns artefatos de requisitos pré-definidos;
    \item \label{oe_ux_facilitada} Viabilizar a criação da \textit{interface}, mencionada no item anterior, de modo a proporcionar uma experiência de usuário satisfatória, visando facilitar o processo da Engenharia de Requisitos, e
    \item \label{oe_resultados} Realizar uma primeira análise dos resultados obtidos, usando como base uma amostra do público alvo.
\end{enumerate}

Adicionalmente aos objetivos específicos listados, pretende-se embasar a pesquisa considerando a literatura especializada, bem como conduzir a pesquisa orientando-se por uma metodologia adequada. Portanto, são realizadas atividade de levantamento bibliográfico e definição de uma metodologia, ambas acordadas no Capítulo de Metodologia (Capítulo \ref{chap:metodologia}).


\section{Organização da Monografia}

\label{ref:organizacao}

A monografia está estruturada conforme os seguintes capítulos:

\begin{itemize}
    \item Capítulo \ref{chap:embasamento_teorico} - Embasamento Teórico: apresenta o que foi encontrado na literatura, no que permeia e oferece apoio, em termos conceituais, ao tema do trabalho;
    \item Capitulo \ref{chap:referencial_tecnologico} - Referencial Tecnológico: descreve as tecnologias que estão sendo utilizadas para o desenvolvimento do trabalho na totalidade;
    \item Capítulo \ref{chap:metodologia} - Metodologia: define como será realizada a condução do trabalho, desde a pesquisa, até a construção da ferramenta e a análise dos resultados;
    \item Capítulo \ref{chap:proposta} - Proposta: acorda outros detalhes, inerentes à proposta deste trabalho, revelando definições, arquitetura, protótipo, além de um esboço inicial do \textit{backlog}\footnote{O termo é explicado na seção \ref{sec:backlog} deste trabalho} do produto, e
    \item Capítulo \ref{chap:consideracoes_finais} - Considerações Finais: apresenta os principais resultados obtidos nesse primeiro momento do desenvolvimento do trabalho.
\end{itemize}