\chapter[Metodologia]{Metodologia}

\label{chap:metodologia}

Neste capítulo, serão apresentados detalhes da metodologia aplicada no desenvolvimento teórico e prático do trabalho. Dentro deste contexto, tem-se, primeiramente, a classificação das atividades de pesquisa conforme a abordagem, natureza, objetivos e procedimentos (seção \ref{sec:met_pesquisa}), seguido da definição da sequência de atividades realizadas (seção \ref{sec:fluxo_atividade}), continuando com a caracterização de aspectos metodológicos adotados no levantamento bibliográfico (seção \ref{sec:levantamento_bibliografico}), seguindo com o panorama do processo desenvolvimento do \textit{software} (seção \ref{sec:metodologia_desenvolvimento}), definindo como será realizada a análise dos resultados obtidos (seção \ref{sec:meto_analise_resultado}), elencando os cronogramas de atividades (seção \ref{sec:cronograma_met}), tanto para o escopo de atividades do TCC1, quanto para o escopo de \textit{atividades} do TCC2, e finalizando com o resumo do capítulo (seção \ref{sec:resumo_metodologia}).

\section{Metodologia de Pesquisa}
\label{sec:met_pesquisa}
De acordo com \citeauthoronline{gerhardt2009metodos} (\citeyear{gerhardt2009metodos}), a Metodologia pode ser classificada como o estudo da organização, dos caminhos a serem trilhados, para se realizar uma pesquisa ou um estudo, ou para produzir ciência. Já a pesquisa científica, refere-se ao resultado de “um inquérito ou exame minucioso, realizado para resolver um problema, recorrendo a procedimentos científicos” \cite{gerhardt2009metodos}.

A partir das definições de \citeauthoronline{gerhardt2009metodos} (\citeyear{gerhardt2009metodos}), esta seção visa categorizar a pesquisa deste trabalho quanto à abordagem, à natureza, aos objetivos e aos procedimentos.

\subsection{Abordagem da Pesquisa}

As pesquisas utilizadas neste trabalho podem ser classificadas como Qualitativas e Quantitativas. A pesquisa qualitativa visa obter uma compreensão mais profunda de um grupo social, de uma organização, entre outros. Portanto, a pesquisa qualitativa, preocupa-se com aspectos da realidade que não podem ser quantificados, concentrando-se em compreender e explicar a dinâmica das relações sociais. Já a pesquisa quantitativa centra-se na objetividade, tendo o objetivo de enfatizar o raciocínio dedutivo, as regras da lógica e as propriedades mensuráveis da experiência humana \cite{gerhardt2009metodos}.

No caso do presente trabalho, pretende-se coletar métricas qualitativas, mais associadas à Satisfação do Usuário quanto ao uso e a pertinência da ferramenta \textit{iFlow}, bem como métricas quantitativas, mais associadas, por exemplo, ao desempenho da ferramenta. Uma típica métrica quantitativa, de interesse deste trabalho, seria a taxa de defeitos encontrados, aferida contando a quantidade de ocorrências desses defeitos, por exemplo, no MVP gerado pela ferramenta.

\subsection{Natureza da Pesquisa}

A pesquisa deste trabalho pode ser classificada como Aplicada. A pesquisa aplicada visa produzir conhecimento para aplicação prática, de modo a resolver problemas específicos. Por outro lado, a pesquisa básica visa gerar conhecimentos novos, sem aplicação prática prevista \cite{gerhardt2009metodos}.

O viés aplicado do trabalho é bastante sensível, dado que a ideia é prover uma ferramenta, para um público alvo, visando mitigar problemas bem específicos do domínio de uma área de interesse da Engenharia de \textit{Software}, no caso, a Engenharia de Requisitos. Adicionalmente, e ainda corroborando com a definição de uma pesquisa aplicada, a ferramenta pretende automatizar algumas atividades, e gerar um MVP.

\subsection{Objetivos da Pesquisa}
Como objetivo, a presente pesquisa classifica-se como Exploratória. A pesquisa exploratória consiste em proporcionar maior familiaridade com o problema, visando torná-lo mais explícito ou formular hipóteses \cite{gil2002elaborar}. Nesse sentido, o trabalho compreende, a partir da identificação de preocupações inerentes à área de Engenharia de Requisitos, explorar soluções que mitiguem, ao menos, tais preocupações, procurando contribuir de alguma forma com o sucesso do \textit{software} produzido, e considerando as boas práticas acordadas na literatura especializada.

\subsection{Procedimentos}

A pesquisa deste trabalho pode ser classificada como Bibliográfica e Pesquisa-Ação. A pesquisa bibliográfica é realizada a partir de um acervo de referências teóricas analisadas e publicadas por meios escritos e tecnológicos, como livros, artigos científicos, \textit{sites}, entre outros. A pesquisa-ação pressupõe a participação planejada do pesquisador na situação problemática a ser investigada, visando diagnosticar um determinado problema sobre uma dada situação, para chegar em um resultado prático \cite{gil2002elaborar}.

Neste trabalho, buscou-se realizar um levantamento bibliográfico, mais bem-apresentado na seção \ref{sec:levantamento_bibliografico}, visando acordar boas práticas e preocupações da área de Engenharia de Requisitos, ambos abordados na ferramenta \textit{iFlow}. Além disso, no intuito de analisar os resultados obtidos, em simultâneo, permitir evoluções na ferramenta com base nesses \textit{feedbacks}, pretende-se usar pesquisa-ação, conforme acordado na seção \ref{sec:meto_analise_resultado}.

\section{Fluxo de Atividades}

\label{sec:fluxo_atividade}
    
Para adequada condução deste trabalho, foram seguidas algumas etapas. Visando melhorar a visualização e evidenciar essas etapas, foi elaborado o modelo, Figura \ref{fig:bpmn_geral}, com base na notação \textit{BPMN}\footnote{Notação gráfica criada pelo grupo conhecido como \textit{BPMI} no ano de 2002 para representação de processos de negócio, disponível em: \href{https://www.devmedia.com.br/introducao-ao-business-process-modeling-notation-bpmn/29892}{Introdução ao BPMN}. Acesso em: 02 abril. 2022}.

\begin{figure}[H]
    \begin{center}
        \caption{Fluxo de Atividades do TCC}
        \label{fig:bpmn_geral}
        \includegraphics[scale=0.3]{figuras/Metodologia/bpmn_geral.png}
        \legend{Fonte: Autores, 2022}
    \end{center}
\end{figure}

\begin{enumerate}
    \item \textbf{Definir Tema}: atividade destinada à escolha do tema para o desenvolvimento do trabalho. Foram apresentados diversos temas de interesse aos orientadores, e após várias discussões, foi escolhido o tema com maior perspectiva de desenvolvimento e interesse entre as partes: a ferramenta \textit{iFlow};
    \\
    Status: Concluída;
    \item \label{item:proposta} \textbf{Formular Proposta}: atividade que visa o fechamento de escopo, bem como contextualizar; identificar questões de pesquisa; estabelecer os objetivos, e conferir justificativas;
    \\
    \textit{Status}: Concluída;
    \\
    Resultado: Capítulo Introdução (\ref{chap:intro});
    \item \textbf{Contextualizar Problema}: atividade que consistiu em obter uma visão mais ampla da área de interesse deste trabalho, no caso, a Engenharia de Requisitos;
    \\
    \textit{Status}: Concluída;
    \\
    Resultado: Capítulo Introdução (\ref{chap:intro});
    \item \textbf{Realizar Levantamento Bibliográfico}: atividade relevante, e que permitiu consultar bases científicas, e investigar a literatura especializada sobre a área de Engenharia de Requisitos a afins. Outros detalhes, inerentes a essa atividade, constam na seção \ref{sec:levantamento_bibliografico};
    \\
    \textit{Status}: Concluída;
    \\
    Resultado: Capítulo Embasamento Teórico (\ref{chap:embasamento_teorico});
    \item \textbf{Definir Referencial Tecnológico}: atividade que compreende a definição dos apoios tecnológicos, no intuito de realizar o trabalho como um todo;
    \\
    \textit{Status}: Concluída;
    \\
    Resultado: Capítulo Referencial Tecnológico (\ref{chap:referencial_tecnologico});
    \item \textbf{Definir Metodologia}: atividade, de suma importância, em que os autores se dedicaram ao desenho de um modelo que viabilizasse a execução do trabalho na totalidade, procurando definir passos a serem cumpridos, e esquematizando-os em atividades;
    \\
    \textit{Status}: Concluída;
    \\
    Resultado: Fluxo de Atividades apresentado nesta seção (\ref{sec:fluxo_atividade});
    \item \textbf{Refinar Proposta}: atividade que visou uma maior compreensão da proposta, procurando refiná-la e detalhá-la;
    \\
    \textit{Status}: Concluída;
    \\
    Resultado: Capítulo Proposta (\ref{chap:proposta});
    \item \textbf{Desenvolver o Protótipo da Ferramenta}: atividade destinada à elaboração de um protótipo de alta fidelidade, cujo objetivo é demonstrar, de forma mais clara, a proposta da ferramenta \textit{iFlow}. O protótipo pode ser visto ainda como uma prova de conceito, em relação à proposta, visto que acorda vários detalhamentos em termos de funcionalidades e fluxos pretendidos na ferramenta;
    \\
    \textit{Status}: Concluída;
    \\
    Resultado: Capítulo Proposta (\ref{chap:proposta});
    \item \textbf{Revisar TCC1}: atividade que procura revisar cada aspecto do TCC, incluindo escrita da monografia, e elaboração de artefatos inerentes ao escopo de trabalho do TCC1. A revisão usa iterações, tanto entre os autores, quanto dos autores com os orientadores;
    \\
    \textit{Status}: Concluída;
    \\
    Resultados: própria Monografia e artefatos em geral;
    
    \item \textbf{Definir \textit{Backlog}}: atividade relevante para iniciar o desenvolvimento da ferramenta. A ideia é elencar de forma clara e objetiva, os principais requisitos da ferramenta;
    \\
    \textit{Status}: A ser realizada;
    \\
    Resultado Esperado: \textit{Backlog} do Produto;
    
    \item \label{item:revision} \textbf{Apresentar TCC1}: atividade destinada à aprovação da proposta, junto à Banca Avaliadora;
    \textit{Status}: Em andamento;
    
    \item \textbf{Corrigir Apontamentos da Banca}: atividade que procura, com base nos \textit{feedbacks} conferidos pela Banca Avaliadora, revisar tanto a escrita da monografia, quanto a elaboração de artefatos;
    \\
    \textit{Status}: A ser realizada;
    \\
    Resultados Esperados: própria Monografia e artefatos em geral evoluídos;
    
    \item \textbf{Desenvolvimento da Ferramenta}: subprocesso que consiste no desenvolvimento da ferramenta. O detalhamento deste subprocesso encontra-se na Figura \ref{fig:bpmn_dev}. As subatividades serão explicadas na seção \ref{sec:met_dev}, na qual é definida a Metodologia de Desenvolvimento;
    \\
    % ARRUMAR AQUI DEPOIS
    \textit{Status}: A ser realizada;
    \\
    Resultado Esperado: Ferramenta \textit{iFlow};
    \item \textbf{Análise dos Resultados}: subprocesso que consiste em aplicar Pesquisa-Ação, visando coletar métricas quantitativas e qualitativas, e realizar evoluções na ferramenta, com base nessas métricas. Pretende-se, nesse processo, validar a ferramenta com apoio de testes e preenchimento de questionários junto ao público alvo. O detalhamento deste subprocesso encontra-se na Figura \ref{fig:bpmn_analise}. As subatividades serão explicadas na seção \ref{sec:meto_analise_resultado}, onde é definida a Metodologia de Análise de Resultados;
    \\
    \textit{Status}: A ser realizada;
    \\
    Resultados Esperados: Pesquisa-Ação e Ferramenta \textit{iFlow} Evoluída;
    \item \textbf{Revisar TCC2}: atividade que procura revisar cada aspecto do TCC, incluindo escrita da monografia, e elaboração de artefatos inerentes ao escopo de trabalho do TCC2. Será utilizado o mesmo processo iterativo, conduzido para o escopo do TCC1, e
    \\
    \textit{Status}: A ser realizada;
    \\
    Resultados Esperados: Monografia e artefatos em geral, em suas versões finais;
    \item \textbf{Apresentar TCC2}: atividade destinada à aprovação do trabalho na totalidade, junto à Banca Avaliadora;
    \\
    \textit{Status}: A ser realizada;
\end{enumerate}

\section{Levantamento Bibliográfico}

\label{sec:levantamento_bibliografico}

O Levantamento Bibliográfico foi realizado de modo a buscar os problemas inerentes à Engenharia de Requisitos. Não foi desenvolvida uma \textit{string} de busca, mas o objetivo foi encontrar fontes confiáveis para o embasamento deste Trabalho de Conclusão de Curso.

Os principais autores e artigos usados tiveram como base referências usadas em disciplinas anteriores, cursadas durante a graduação. As palavras-chave utilizadas para guiar as pesquisas foram as mesmas usadas para as definições contidas no Embasamento Teórico (Capítulo \ref{chap:embasamento_teorico}), tais como: Engenharia de Requisitos, Elicitação, Modelagem, \textit{House of Quality}, entre outros.

Vale ressaltar que os termos correspondentes foram utilizados na língua inglesa para haver uma maior quantidade de referências no contexto proposto. Obtivemos um total de 27 referências, dentre artigos, livros e sítios que podem ser melhor visualizados no \href{https://unb-rogerio.notion.site/Refer-ncias-c31bfce5dba543619a9d0913f4ba9a00}{Documento de Referências} usado para a centralização e manipulação da bibliografia do trabalho. Destaca-se, também, que essas foram as referências levantadas em um primeiro momento, mas que a quantidade e as citações completas podem ser visualizadas nas Referências ao final do trabalho.

Para a seleção das referências, para a pesquisa bibliográfica, os seguintes itens foram utilizados como critérios de exclusão:

\begin{itemize}
    \item Excluem-se artigos em idiomas diferentes de inglês e português;
    \item Excluem-se artigos que não falam a respeito dos processos de Engenharia de Requisitos;
    \item Excluem-se artigos que não se aplicam ao contexto de \textit{software};
    \item Excluem-se artigos que possuem as definições necessárias;
    \item Incluem-se artigos que falam sobre a Engenharia de Requisitos e seus processos;
    \item Incluem-se artigos que falam sobre Elicitação, Modelagem, Análise, e Rastreabilidade;
    \item Incluem-se artigos que falam sobre \textit{House of Quality}, e
    \item Incluem-se artigos que falem sobre as Estruturas de Dados.
\end{itemize}

\section{Metodologia de Desenvolvimento}

\label{sec:metodologia_desenvolvimento}

No que tange a metodologia voltada ao desenvolvimento prático da ferramenta, escolheu-se, por afinidade da dupla, a combinação de duas metodologias, \textit{Scrum} \& \textit{Kanban}, as quais serão adaptadas em função do perfil do trabalho.

O \textit{Scrum} é uma metodologia ágil para desenvolvimento de produtos complexos que mudam os requisitos rapidamente. O seu desenvolvimento é dado por uma série de iterações chamadas \textit{Sprint}, que no caso desse projeto, serão semanais. A cada \textit{Sprint} é realizada uma reunião de planejamento (\textit{Sprint Planning}) e de revisão do trabalho (\textit{Sprint Review \& Restrospective}), com foco na melhoria contínua do processo e iteração entre as pessoas. Para promover a transparência e o alinhamento do trabalho, serão utilizados dois artefatos, o \textit{Backlog} do Produto (\textit{Product Backlog}) e o \textit{Backlog} da \textit{Sprint} (\textit{Sprint Backlog}), nos quais  são descritos, respectivamente, o
escopo do produto e o escopo da \textit{Sprint} \cite{carolipaulo2021}.

O \textit{Kanban} é uma metodologia que visa um alinhamento mais claro de uma equipe, além de dar visibilidade ao que está sendo elaborado, proporcionando um ambiente favorável à comunicação, com o menor ruído possível. Além disso, garante uma carga de trabalho adequada, considerando a capacidade produtiva da equipe. Um ponto essencial dessa metodologia é o quadro que deixa as tarefas e seus \textit{status} visíveis para todo o time, pontuando em qual parte do fluxo de trabalho cada tarefa se encontra \cite{K_Condensed}.

\subsection{Processo de Desenvolvimento}

\label{sec:met_dev}

O processo de desenvolvimento escolhido pode ser visualizado na Figura \ref{fig:bpmn_dev}, elaborado com base na notação \textit{BPMN},  a partir do detalhamento do subprocesso \textbf{Desenvolvimento da Ferramenta}, apresentado na Figura \ref{fig:bpmn_geral}, consistindo das seguintes etapas:

\begin{figure}[H]
    \begin{center}
        \caption{Fluxo de Atividades do Subprocesso Desenvolvimento da Ferramenta \textit{iFlow}}
        \label{fig:bpmn_dev}
        \includegraphics[scale=0.23]{figuras/Metodologia/bpmn_dev.png}
        \legend{Fonte: Autores, 2022.}
    \end{center}
\end{figure}

\begin{enumerate}
    \item \textbf{Realizar \textit{Sprint Planning}}: a partir do \textit{Product Backlog}, as atividades são selecionadas consoante a sua prioridade e pontos definidos para execução da mesma. As tarefas selecionadas consideram a capacidade produtiva dos autores na \textit{Sprint} e a gestão de riscos para a entrega do produto. Ao final, é esperado o \textit{Sprint Backlog};
    \item \textbf{Selecionar o item do \textit{Sprint Backlog}}: a partir do \textit{Sprint Backlog}, as tarefas são selecionadas para serem executadas pelos autores;
    \item \textbf{Fazer a tarefa}: consiste em desenvolver a tarefa selecionada;
    \item \textbf{Revisar tarefa}: consiste no processo de revisão por parte, dentre os autores, do membro que não atuou no desenvolvimento diretamente, para verificar se os padrões de desenvolvimento da comunidade foram seguidos, e se atende ao que foi definido no \textit{Product Backlog};
    \item \textbf{Realizar \textit{Sprint Review \& Restrospective}}: com a conclusão da \textit{Sprint}, essa atividade elenca os pontos positivos e negativos da \textit{sprint}, além de revisar as atividades entregues.
\end{enumerate}


\section{Metodologia de Análise de Resultados}

\label{sec:meto_analise_resultado}

Como definido na Seção \ref{sec:met_pesquisa}, o método orientado para o processo de análise dos resultados do trabalho será a Pesquisa-Ação. De acordo com \citeauthoronline{gil2002elaborar} (\citeyear{gil2002elaborar}), a pesquisa-ação ocorre com muitas oscilações entre as fases, determinada pela dinâmica do grupo de pesquisadores em seu relacionamento com a situação pesquisada. Além disso, a pesquisa-ação apresenta uma série de ações desordenadas no tempo, considerando os seguintes passos: a) fase exploratória; b) formulação do problema; c) construção de hipóteses; d) realização do seminário; e) seleção da amostra; f) coleta de dados; g) análise e interpretação dos dados; h) elaboração do plano de ação; e i) divulgação dos resultados.

Já conferindo uma versão adaptada do protocolo, que pode ser mais bem visualizada na Figura \ref{fig:bpmn_analise}, a pesquisa-ação a ser utilizada neste trabalho será conduzida com base nas seguintes etapas:

\begin{figure}[H]
    \begin{center}
        \caption{Fluxo de Atividades do Subprocesso Análise dos Resultados da Ferramenta \textit{iFlow}}
        \label{fig:bpmn_analise}
        \includegraphics[scale=0.28]{figuras/Metodologia/bpmn_analise.png}
        \legend{Fonte: Autores, 2022.}
    \end{center}
\end{figure}

\begin{itemize}
    \item \textbf{Coletar dados}: nesta etapa, realiza-se a coleta de dados quantitativos por meio da ferramenta desenvolvida, e qualitativos através da interação dos usuários com a ferramenta;
    \item \textbf{Analisar e interpretar dos dados}: nesta etapa, busca-se examinar os dados coletados e, em seguida, realizar a explanação dos resultados adquiridos;
    \item \textbf{Elaborar plano de ação}: pretende-se elaborar um plano de ação que visa mitigar os problemas encontrados a partir dos dados analisados, e 
    \item \textbf{Divulgar resultados}: para validação, os resultados obtidos e o plano de ação definido serão documentados e validados com a banca, bem como junto ao público alvo, no intuito de compreender sobre as contribuições da ferramenta no escopo de atuação da mesma.
\end{itemize}

Dentre as investigações, pretende-se ter uma iteração mais voltada a testes com os \textit{stakeholders} que avaliam a experiência do usuário na ferramenta, de forma a validar se a ferramenta cumpre com os \textbf{objetivos específicos} \ref{oe_guiar_usuario} e \ref{oe_ux_facilitada}. Pretende-se também realizar testes que busquem validar se os requisitos propostos foram cumpridos e, principalmente, se o \textbf{objetivo específico} \ref{oe_mvp} foi satisfeito. Há ainda a possibilidade de realizar a coleta de métricas quantitativas, sendo assim, havendo a necessidade de pelo menos uma iteração nesse sentido.

\section{Cronogramas}

\label{sec:cronograma_met}

As Tabelas \ref{tab:cronograma_tcc1} e \ref{tab:cronograma_tcc2} apresentam os cronogramas do TCC1 e do TCC2, respectivamente. Nas tabelas, as atividades são descritas de acordo com suas datas de implementação, baseando-se no fluxo de atividades descrito na Seção \ref{sec:fluxo_atividade}, e considerando uma temporalidade contínua e não baseada apenas em período letivo.

\begin{table}[H]
    \centering
    \caption{Cronograma de Atividades do TCC1}
    \scalebox{0.8}{%
    \begin{tabular}{l*{4}{c}r}
        \hline
        Atividade & Jan/2022 & Fev/2022 & Mar/2022 & Abr/2022 & Mai/2022 \\
        \hline
        Definir Tema & X & & & & \\
        Formular Proposta & & X & & & \\
        Contextualizar Problema & & X & & & \\
        Realizar Levantamento Bibliográfico & & & X & & \\
        Definir Referencial Tecnológico & & & X & & \\
        Definir Metodologia & & & X & X & \\
        Refinar Proposta & & & X & X & \\
        Desenvolver o Protótipo da Ferramenta & & & & X & \\
        Revisar TCC1 & & & & X & \\
        Definir \textit{Backlog} & & & & X & \\
        Apresentar TCC1 & & & & & X \\
        \hline
    \end{tabular}}
    \legend{Fonte: Autores, 2022.}
    \label{tab:cronograma_tcc1}
\end{table}

\begin{table}[H]
    \centering
    \caption{Cronograma de Atividades do TCC2}
    \scalebox{0.8}{%
    \begin{tabular}{l*{5}{c}r}
        \hline
        Atividade/Subprocesso & Mai/2022 & Jun/2022 & Jul/2022 & Ago/2022 & Set/2022 \\
        \hline
        Corrigir Apontamentos da Banca & X & & & & \\
        Desenvolvimento da Ferramenta & X & X & X & X & \\
        Análise dos Resultados & & & & X & X \\
        Revisar TCC2 & & & & & X \\
        Apresentar TCC2 & & & & & X \\
        \hline
    \end{tabular}}
    \legend{Fonte: Autores, 2022.}
    \label{tab:cronograma_tcc2}
\end{table}

\section{Resumo do Capítulo}
\label{sec:resumo_metodologia}
Neste capítulo, foram apresentadas as escolhas metodológicas definidas para o desenvolvimento deste Trabalho de Conclusão de Curso. Desta forma, foram evidenciadas as metodologias da fase de pesquisa, de desenvolvimento da proposta, e de análise dos resultados, a fim de identificar e categorizar os pontos mais importantes em cada uma delas. A seguir, visando retratar as etapas de desenvolvimento do trabalho, foram apresentados os fluxos de atividades que orientam o desenvolvimento do trabalho com as respectivas descrições correspondentes das atividades ilustradas nesses fluxos. Por fim, os cronogramas das duas fases de realização do trabalho (TCC1 e TCC2) foram apresentados.