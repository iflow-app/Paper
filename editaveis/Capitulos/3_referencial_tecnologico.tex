\chapter[Referencial Tecnológico]{Referencial Tecnológico}

\label{chap:referencial_tecnologico}

Este capítulo apresenta as principais tecnologias que conferem apoio ao desenvolvimento deste trabalho na totalidade. As escolhas tecnológicas fundamentaram-se no uso de recursos gratuitos ou de código aberto; na afinidade dos autores, e no conhecimento prévio, considerando as recomendações da comunidade de \textit{software} diante das necessidades de projeto apresentadas. A intenção foi minimizar os riscos quanto ao desenvolvimento da ferramenta desenvolvida. O capítulo está organizado em três partes, sendo: Tecnologias de Apoio à Aplicação (seção \ref{tec_apoio_app}), Tecnologias de Apoio à Pesquisa e à Comunicação (seção \ref{tec_apoio_pesquisa_comunicacao}) e o Resumo do Capítulo (seção \ref{sec_tec_resumo}).


\section{Tecnologias de Apoio à Aplicação}

\label{tec_apoio_app}

\subsection{TypeScript}
O \textit{TypeScript} (TS) é uma linguagem dinâmica, interpretada e multiparadigma, que confere recursos aos estilos orientados a objetos, imperativos e declarativos \cite{typescript}, além de ser fortemente tipada. A linguagem foi utilizada a partir do emprego do \textit{NodeJS} e do \textit{ReactJS} como ferramentas de apoio.

\subsection{ReactJS}
O \textit{ReactJS} é uma biblioteca \textit{JavaScript} visando criar \textit{interfaces} de usuário \cite{reactjs}. O \textit{ReactJS} foi escolhido devido ao seu bom desempenho com virtual DOM (\textit{Document Object Model}), e pela facilidade de criação de componentes que ele fornece. Além disso, o \textit{ReactJS} utiliza visualizações declarativas, que torna o código mais previsível e fácil de depurar.

\subsection{NodeJS}
O \textit{NodeJS} é uma plataforma projetada para o desenvolvimento de aplicações escaláveis de rede, sendo um ambiente de execução de código \textit{JavaScript} \cite{nodejs}. O NodeJS foi utilizado para a implementação da \textit{API} que é a comunicação entre as camadas da ferramenta.

\subsection{GitHub}
O \textit{GitHub} é uma plataforma que auxilia no ciclo de desenvolvimento de uma aplicação, hospedando o código-fonte \cite{github}. A plataforma foi utilizada para armazenar e centralizar o código-fonte da ferramenta.

\subsection{Git}
O \textit{Git} é uma ferramenta de controle de versão gratuita e de código. aberto \cite{git}. Foi utilizado para realizar o versionamento do código-fonte da ferramenta.

\subsection{Docker}
O \textit{Docker} é uma plataforma para a criação, execução e publicação de contêineres \cite{docker}. Essa plataforma foi utilizada para a gerência de configuração da ferramenta, objetivando facilitar e otimizar as atividades de desenvolvimento e de implantação com a utilização dos contêineres.

\subsection{Docker Compose}
O \textit{Docker Compose} é uma ferramenta para definir e executar aplicativos \textit{Docker} de vários contêineres. Com o Compose, faz-se uso de um arquivo \textit{YAML} para configurar os serviços da aplicação \cite{docker-compose}. O \textit{Docker Compose} foi utilizado para a orquestração dos contêineres dos serviços utilizados pela ferramenta.

\subsection{Trello}
O Trello é uma ferramenta de gerenciamento de projetos que utiliza um esquema de listas, cartões e quadros para organizar atividades dentro de um projeto \cite{trello}. A ferramenta foi utilizada para a criação de um quadro \textit{Kanban}, permitindo que haja o acompanhamento das atividades inerentes ao projeto.

\subsection{Figma}
Figma é um editor de gráficos vetoriais e, ao mesmo tempo, uma ferramenta que pode ser utilizada para prototipagem \cite{figma}. Optou-se por essa ferramenta, uma vez que, a mesma viabilizou a modelagem de um protótipo de alta fidelidade do produto, oriundo deste trabalho.

\section{Tecnologias de Apoio à Pesquisa e à Comunicação}

\label{tec_apoio_pesquisa_comunicacao}

\subsection{Google Drive}
O \textit{Google Drive} é uma plataforma criada pelo \textit{Google}, e consiste em armazenar, compartilhar e colaborar arquivos e pastas \cite{drive}. O \textit{Google Drive} foi utilizado para a elaboração de planilhas e documentos colaborativos referentes ao projeto.

\subsection{LaTeX3}
O \textit{LaTeX3} é um sistema de preparação de documentos para composição tipográfica de alta qualidade \cite{latex}. O \textit{LaTeX3} foi utilizado pelo fato de auxiliar na formatação do documento.

\subsection{Overleaf Community Edition}
O \textit{Overleaf} é uma ferramenta \textit{online} de escrita em \textit{LaTeX} e publicação colaborativa para a elaboração de documentos técnicos e científicos \cite{overleaf}. O \textit{Overleaf} foi utilizado para a produção e a edição do texto científico.

\subsection{Telegram}
O \textit{Telegram} é um aplicativo, disponível nas versões \textit{mobile} e \textit{web}, para o compartilhamento de mensagens. Além de ter suporte ao envio de mensagens de texto, voz, mídia e documentos, os usuários podem fazer chamadas de vídeo e ligações \cite{telegram}. O aplicativo foi utilizado para o compartilhamento de arquivos, \textit{links}, reuniões, alertas e discussões imediatas.

\subsection{Discord}
O \textit{Discord} é uma plataforma de comunicação que permite a efetuação de chamadas e a realização de videochamadas \cite{discord}. A plataforma foi muito utilizada para a realização de reuniões remotas entre os desenvolvedores do projeto.

\subsection{Notion}
O \textit{Notion} é uma plataforma utilizada para criar sistemas de gerenciamento de conhecimento, gerenciamento de dados, anotações, gerenciamento de projetos, entre outros \cite{notion}. A ferramenta foi utilizada para haver organização e separação das atividades, além de salvar os artigos utilizados como base para o desenvolvimento do projeto.

\section{Resumo do Capítulo}

\label{sec_tec_resumo}

Neste capítulo, foram apresentadas as principais tecnologias que auxiliaram no desenvolvimento da ferramenta de apoio ao processo de Engenharia de Requisitos, \textit{iFlow}, bem como as ferramentas que foram utilizadas para dar apoio à pesquisa e à comunicação.

As ferramentas de apoio ao desenvolvimento da ferramenta foram dispostas de modo a descrever como foram utilizadas para contribuir desde o gerenciamento do código-fonte, incluindo gerência de configuração e evolução, até o desenvolvimento do sistema, envolvendo as camadas de processamento de dados (\textit{back-end}) e de \textit{interface} com o usuário (\textit{front-end}).

Além disso, as ferramentas de apoio à pesquisa e à comunicação foram descritas, de modo a indicar como foram utilizadas para o processo de pesquisa, referenciação e escrita. Adicionalmente, têm-se as ferramentas para facilitar a comunicação entre as partes desenvolvedoras do projeto, completando, assim, a apresentação do suporte tecnológico necessário na elaboração deste trabalho.