\chapter[Conclusão]{Conclusão}

\label{chap:consideracoes_finais}

Este capítulo apresenta as considerações finais das atividades relacionadas ao Trabalho de Conclusão de Curso, bem como os resultados alcançados durante a condução, a elaboração e o desenvolvimento. Dessa forma, em um primeiro momento, é apresentado o andamento geral do trabalho (seção \ref{sec:andamento_do_trabalho}), desde a Introdução até a Apresentação para a banca (TCC2). Na sequência, constam as respostas para as \hyperref[ref:questao_pesquisa]{questões de pesquisa} levantadas (seção \ref{sec:perguntas_respondidas}). Há ainda um detalhamento sobre o cumprimento dos \hyperref[ref:objetivos]{objetivos} propostos (seção \ref{sec:objetivos_concluido}). Por fim, são elencadas as principais contribuições e fragilidades da \hyperref[chap:proposta]{Ferramenta \textit{iFlow}}, dando um panorama de melhorias para trabalhos futuros.

\section{\textit{Status} do Trabalho}

\label{sec:andamento_do_trabalho}

Na primeira etapa deste trabalho, focou-se no desenvolvimento das atividades que fundamentavam e sistematizavam a proposta de projeto em relação à criação de uma ferramenta, capaz de conferir apoio e semiautomatizações às etapas relacionadas ao processo de Engenharia de Requisitos. Nesse sentido, a Tabela \ref{tab:atividades_tcc_1} mostra o andamento das atividades da fase inicial do trabalho, desde a definição do tema até à apresentação aos membros da banca (TCC1), sendo todas tarefas já concluídas.

\begin{table}[H]
    \centering
    \caption{Andamento das atividades do TCC1}
    \scalebox{0.9}{%
    \begin{tabular}{l*{2}{c}r}
        \hline
        Atividade & Andamento \\
        \hline
        Definir Tema & Concluída \\
        Formular Proposta & Concluída \\
        Contextualizar Problema & Concluída \\
        Realizar Levantamento Bibliográfico & Concluída \\
        Definir Referencial Tecnológico & Concluída \\
        Definir Metodologia & Concluída \\
        Refinar Proposta & Concluída \\
        Desenvolver o Protótipo da Ferramenta & Concluída \\
        Revisar TCC1 & Concluída \\
        Definir \textit{Backlog} & Concluída \\
        Apresentar TCC1 & Concluída \\
        \hline
    \end{tabular}}
    \legend{Fonte: Autores, 2022.}
    \label{tab:atividades_tcc_1}
\end{table}

Em um segundo momento, a condução do trabalho deu-se de forma mais prática, para desenvolver a ferramenta \hyperref[chap:proposta]{\textit{iFlow}}; levantar dados para avaliar a usabilidade da  ferramenta, e divulgar os resultados, documentando-os nesta monografia. Nesse sentido, a Tabela \ref{tab:atividades_tcc_2} mostra o \textit{status} de todas as atividades conduzidas nesse contexto, sendo todas já concluídas (exceto a apresentação à banca, que ocorrerá em breve).

\begin{table}[H]
    \centering
    \caption{Andamento das atividades do TCC2}
    \scalebox{0.8}{%
    \begin{tabular}{l*{2}{c}r}
        \hline
        Atividade & Andamento \\
        \hline
        Corrigir Apontamentos da Banca & Concluída \\
        Desenvolvimento da Ferramenta & Concluída \\
        Coleta de Dados & Concluída \\
        Análise dos Resultados & Concluída \\
        Revisar TCC2 & Concluída \\
        Apresentar TCC2 & A fazer \\
        \hline
    \end{tabular}}
    \legend{Fonte: Autores, 2022.}
    \label{tab:atividades_tcc_2}
\end{table}

\section{\textit{Status} das Questões de Pesquisa}

\label{sec:perguntas_respondidas}

Para a condução deste estudo, foram levantadas algumas indagações que permearam os passos e as decisões tomadas. A partir de todos os expostos e evidências presentes nesse Trabalho de Conclusão de Curso, conclui-se que:

\begin{enumerate}
    \item \textbf{Como podemos desenvolver uma ferramenta que apoie o Engenheiro de Requisitos na sintetização de um produto de \textit{software} abstrato para um que seja concreto?}: diante dos resultados apresentados nos Capítulos \hyperref[chap:proposta]{\textit{iFlow}} e \hyperref[sec:ana_int_dados]{Análise e Interpretação dos Dados}, e das respostas obtidas pelo Questionário e pelo Teste de Usabilidade, pode-se concluir que o \textit{iFlow} aponta em direção de solução. Porém, há necessidade de melhorias, sendo essas oportunidades para trabalhos futuros. Nesse sentido, sabe-se \textit{iFlow} é orientado pelos processos da Engenharia de Requisitos de forma simplificada e intuitiva, além de se apoiar na literatura especializada para que se tenha um MVP, na sua versão preliminar, como resultado. Além disso, o \textit{iFlow} foi pensado para que os Engenheiros de Requisitos façam todas as etapas do processo para evitar retrabalhos e desperdícios de tempo.
    \item \textbf{Como podemos facilitar o desenvolvimento criativo do Engenheiro de Requisitos, de tal forma que a rastreabilidade das fontes dos requisitos sejam mantidas?}: para prover um desenvolvimento criativo e manter a rastreabilidade dos requisitos, o \textit{iFlow} foi desenvolvido de tal forma que o usuário consiga criar seus próprios artefatos e, posteriormente, extrair os requisitos para cada um, mantendo tudo salvo. Dessa forma, o usuário consegue acessar e visualizar os rastros das fontes destes requisitos sem dificuldades, provendo uma aplicação que não limita a criatividade do seu usuário por definições limitantes de funcionalidades.
    \item \textbf{Tendo posse dos requisitos amadurecidos e sua importância para o produto de \textit{software}, como podemos correlacionar esses dados para gerar um possível \textit{Minimum Viable Product} (MVP)?}: a ferramenta \textit{iFlow} foi desenvolvida com o objetivo de gerar um possível MVP, na sua versão preliminar, fazendo a correlação dos dados provenientes de todo o processo de Engenharia de Requisitos proposto, conforme descrito no capítulo de \hyperref[chap:embasamento_teorico]{Embasamento Teórico}. Dessa forma, o \textit{iFlow} apoia o usuário na realização das etapas, passo a passo, para que, no final, com os requisitos detalhados e etapas finalizadas, ele consiga gerar um possível MVP para a sua aplicação.
\end{enumerate}

\section{Objetivos Concluídos}
\label{sec:objetivos_concluido}
Retomando os objetivos específicos, apresentados no capítulo de \hyperref[chap:intro]{Introdução}, descritos como:

\begin{itemize}
    \item \textbf{Automatizar a \textit{hiperlinkagem} entre os artefatos de requisitos}:\\
    \textit{Status}: Visão preliminar proposta, cabendo maior aprimoramento na renderização desses dados na interface. \\
    \item \textbf{Viabilizar a geração do MVP (versão preliminar) com base nos artefatos gerados}:\\
    \textit{Status}: Visão preliminar proposta, cabendo ainda pequenos ajustes, com base no andamento do projeto, em termos evolutivos e trabalhos futuros.
    \item \textbf{Prover uma \textit{interface} para a ferramenta que guie o usuário no processo de criação de alguns artefatos de requisitos pré-definidos}:\\
    \textbf{Status}: Visão preliminar proposta, cabendo ainda maior aprimoramento em alguns aspectos mencionados na seção de \hyperref[sec:ana_int_dados]{Análise e Interpretação dos Dados}, para que se tenha uma maior facilidade em se realizar o que foi proposto em cada etapa do processo.
    \item  \textbf{Viabilizar a criação da \textit{interface}, mencionada no item anterior, de modo a proporcionar uma experiência de usuário satisfatória, visando facilitar o processo da Engenharia de Requisitos}:\\
    \textit{Status}: Visão preliminar proposta, cabendo ainda pequenos ajustes de interface para que se tenha uma maior usabilidade, com base no andamento do projeto, em termos evolutivos e com trabalhos futuros.
    \item  \textbf{Realizar uma primeira análise dos resultados obtidos, usando como base uma amostra do público alvo}:\\
    \textit{Status}: Visão preliminar proposta, podendo ser visualizada na seção \hyperref[sec:ana_int_dados]{Análise e Interpretação dos Dados}.
\end{itemize}

Foi possível alcançar grande parte dos objetivos específicos, conforme descritos na seção \hyperref[sec:objetivos_especificos]{Objetivos Específicos}. Com base nesse embasamento teórico, foi possível propor uma ferramenta capaz de gerar um Produto Mínimo Viável, na sua versão preliminar, a partir das etapas mais relevantes dentro do processo de Engenharia de Requisitos.

Realizou-se uma Pesquisa-Ação para validação da usabilidade do \textit{iFlow} e, para isso, foram elaborados um questionário e um teste de usabilidade na ferramenta \href{https://usabilityhub.com/}{\textit{UsabilityHub}}, para avaliar o uso da ferramenta desenvolvida. As validações podem ser vistas no Capítulo \hyperref[chap:analise_resultados]{Análise dos Resultados}.

\section{Considerações do \textit{iFlow}}

\subsection{Contribuições}
O \textit{iFlow} é uma ferramenta que centraliza todo o processo de desenvolvimento da Engenharia de Requisitos, facilitando a \textit{linkagem} entre artefatos e requisitos e a consolidação dos rastros entre ambos. Confere ao usuário a liberdade de adicionar artefatos produzidos externamente e oferece uma interface capaz de extrair os requisitos concomitantemente.

Além de ter o escopo de consolidar artefatos produzidos, tanto na \hyperref[sec:pre-rastreabilidade]{pré-rastreabilidade} como \hyperref[sec:elicitacao]{elicitação}, o \textit{iFlow} conseguiu prover telas, de implementação não trivial, com interfaces simplificadas, capazes de facilitar a montagem de artefatos complexos, como o \hyperref[sec:backlog]{\textit{Backlog}} e o \hyperref[sec:nfr]{NFR}. Ainda que sejam necessários ajustes, como foi revelado no Capítulo de \hyperref[chap:analise_resultados]{Análise dos Resultados}, as telas tiveram alta taxa de acertos e \textit{feedbacks} dos usuários.

Ressalta-se, sendo esse um ponto de significativa relevância, que a ferramenta provê uma forma de relacionar requisitos funcionais e não funcionais, conferindo um olhar mais qualitativo a produtos de \textit{software}. A partir desses relacionamentos, o \textit{iFlow} ainda sugere um Produto Mínimo Viável, em sua forma preliminar, que viabiliza uma versão do produto, agregando valor ao usuário.

\subsection{Fragilidades}

Dentre os pontos que necessitam de maior atenção na ferramenta, há necessidade de implementação das funcionalidades que não fizeram parte dessa primeira versão, como já mencionado na seção \hyperref[sub:corte_de_escopo]{Ajustes de Escopo}. Com isso, sendo possível uma automatização completa das etapas propostas da Engenharia de Requisitos.

Existe a demanda também de melhorar a correlação entre os requisitos funcionais e não funcionais, para poder ser considerado no momento da geração do Produto Mínimo Viável.

Por fim, vale ressaltar que, ainda que o \hyperref[sec:prototipo_de_alta_fidelidade]{Protótipo de Alta Fidelidade} tenha sido desenvolvido com bastante cuidado, buscando facilitar o trabalho do usuário, alguns pontos de melhoria foram levantados na \hyperref[chap:analise_resultados]{Análise dos Resultados} e precisam de atenção.

\subsection{Trabalhos Futuros}

Diante do exposto, alguns pontos de melhoria são necessários para que a ferramenta atenda de forma mais adequada a Engenharia de Requisitos, sendo eles:

\begin{enumerate}
    \item Melhorar os aspectos de usabilidade e de interface para que os processos na aplicação fiquem mais intuitivos;
    \item Evoluir o NFR para aumentar o número de aspectos qualitativos;
    \item Desenvolver as funcionalidades que foram desconsideradas, ao se ajustar o escopo (seção \ref{sub:corte_de_escopo});
    \item Realizar e automatizar a \textit{hyperlinkagem} dos léxicos no contexto da ferramenta, e
    \item Aperfeiçoar o algoritmo de menor caminho para que ele possa considerar aspectos qualitativos na geração do MVP.
\end{enumerate}