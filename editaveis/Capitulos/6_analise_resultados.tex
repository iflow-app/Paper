\chapter[Análise de Resultados]{Análise de Resultados}

\label{chap:analise_resultados}
Este capítulo tem por objetivo apresentar os resultados deste trabalho, bem como descrever as atividades de análise aplicadas sobre a ferramenta \textit{iFlow}, Capítulo \ref{chap:proposta}, e acordar sobre as ações de melhoria identificadas. A Análise dos Resultados foi conduzida com base na Metodologia de Análise de Resultados (Seção \ref{sec:meto_analise_resultado}), já apresentada anteriormente. Essa estabelece um protocolo de Pesquisa-Ação, o qual compreende as fases de Coleta de Dados (Seção \ref{sec:coleta_de_dados}), Análise e Interpretação dos Dados (Seção \ref{sec:ana_int_dados}), Elaboração do Plano de Ação (Seção \ref{sec:plano_de_acao}), e Divulgação dos Resultados (Seção \ref{sec:divulgacao_resultados}), conforme apresentado na sequência. Por fim, é apresentado o resumo do capítulo (Seção \ref{sec:consideracoes_finais_analise}).

\section{Fases da Pesquisa-Ação}
Conforme descrito na Metodologia de Análise de Resultados (Seção \ref{sec:meto_analise_resultado}), este trabalho segue as fases da Pesquisa-Ação. Em um primeiro momento, consta a fase de Coleta de Dados, sendo essa breve. Logo em seguida, tem-se a fase de Análise e Interpretação dos Dados, onde os dados são coletados e analisados de forma, predominantemente, qualitativa. Seguindo para a Elaboração do Plano de Ação, é realizado um planejamento para solucionar/mitigar erros e imprecisões apontadas pela fase anterior. Por fim, tem-se a fase de Divulgação de Resultados, concluindo o protocolo de Pesquisa-Ação.

\section{Coleta de Dados}
\label{sec:coleta_de_dados}
Nesta fase, inicia-se o processo da validação da ferramenta \textit{iFlow} que foi desenvolvida ao longo deste Trabalho de Conclusão de Curso, com a identificação do problema e a descrição do contexto em que esse se insere. Problema: \textit{\textbf{Como podemos desenvolver uma ferramenta que consiga apoiar o Engenheiro de Requisitos na sintetização de um Produto Mínimo Viável de \textit{software} em sua versão preliminar?}} Contexto: um dos principais aspectos levantados nos objetivos específicos foi a preocupação da qualidade da interação do usuário com a ferramenta (Seção \ref{sec:objetivos_especificos}), evitando retrabalhos e desperdício de tempo.

Para isso, a coleta de dados foi feita visando coletar, do nosso público alvo, dados com relação à interação e a satisfação do uso da ferramenta \textit{iFlow}. Para tanto, a aplicação dos testes deu-se sem treinamento prévio, justamente para se ter dados mais precisos na intuitividade da aplicação proposta. 

\section{Análise e Interpretação dos Dados}
\label{sec:ana_int_dados}
Nesta fase, foi realizada a coleta de \textit{feedbacks}, visando validar a ferramenta \textit{iFlow} no processo de Engenharia de Requisitos. Os participantes são atuantes na área de \textit{software} e requisitos. Esse público possui experiência prévia dos conceitos inerentes ao escopo deste trabalho, podendo validar o uso da ferramenta \textit{iFlow} com mais tranquilidade.

O formulário foi criado pela plataforma do \textit{Google Forms}, ficou disponível para respostas durante 5 dias corridos, e obteve um total de 40 respostas. Vale ressaltar que todos os usuários que enviaram a pesquisa, concordaram com a divulgação dos dados no contexto deste Trabalho de Conclusão de Curso de forma anônima, com termo de consentimento de conhecimento de todos.

\subsection{Questionário}
Para validar o \textit{iFlow}, foi criado um questionário com algumas perguntas, obrigatórias ou não, além de um teste de usabilidade, de modo a entender a experiência dos usuários. Essa seção apresenta as perguntas inseridas no questionário e, em seguida, as respostas obtidas, por meio de Figuras com os gráficos das respostas. As respostas de escolha múltipla usaram a escala de 1(um) a 5(cinco), validando dentro de uma escala qualitativa indo de péssimo(1) a ótimo(5). 

O Questionário foi dividido em três seções, sendo elas: \textit{House of Quality}, NFR Framework (\textit{Non-Functional Requirements}) e \textit{Backlog} do Produto, em que o objetivo foi de validar a usabilidade dos 3 módulos considerados primordiais para o funcionamento da ferramenta \textit{iFlow}.

%%%%%%%%%%%%%%%%%%%%%%%%%%%% HOUSE OF QUALITY %%%%%%%%%%%%%%%%%%%%%%%%%%%%%%%%%%%%%%%%%%

\subsubsection{\textit{House of Quality}}
A seção do \textit{House of Quality}, no questionário, dividiu-se da seguinte forma:

\begin{enumerate}
    \item Uma breve seção explicando como a \textit{House of Quality} se aplica no \textit{iFlow},
    \item Uma pequena descrição e um \textit{hyperlink} do teste de usabilidade que foi realizado pelo usuário, e
    \item As perguntas referentes ao teste de usabilidade.
\end{enumerate}

A Figura \ref{fig:hq_secao} ilustra a primeira seção do questionário, onde é retratada a \textit{House of Quality}.

\begin{figure}[H]
    \begin{center}
        \caption{{Seção do \textit{House of Quality} no Questionário desenvolvido}}
        \label{fig:hq_secao}
        \includegraphics[scale=0.5]{figuras/questionario/house-of-quality-section-1.png}
        \legend{Fonte: Autores, 2022.}
    \end{center}
\end{figure}

Dessa forma, após a realização do teste por parte do usuário, foram indagadas as seguintes perguntas:

\begin{itemize}
    \item Quão fácil você achou de realizar a tarefa?
    \item Quão importante foi o processo de arrastar os requisitos funcionais para dentro dos não funcionais?
    \item O que você acredita que poderia ser melhorado nesse processo?
\end{itemize}


\begin{figure}[]
  \begin{center}
      \caption{{Resultados da Pergunta 1: Quão fácil você achou de realizar a tarefa?}}
      \label{fig:hq_respostas_1}
      \includegraphics[scale=0.65]{figuras/questionario/resultados-hq-1.png}
      \legend{Fonte: Autores, 2022.}
  \end{center}
\end{figure}

A Figura \ref{fig:hq_respostas_1} representa as respostas acerca da facilidade em se realizar a tarefa na etapa da \textit{House of Quality}. Conclui-se que grande parte dos usuários conseguiram realizar a tarefa de modo fácil, tendo \textbf{77,5\%} das respostas finais nas notas 4 e 5.

\begin{figure}[]
  \begin{center}
      \caption{{Resultados da Pergunta 2: Quão importante foi o processo de arrastar os requisitos funcionais para dentro dos não funcionais?}}
      \label{fig:hq_respostas_2}
      \includegraphics[scale=0.65]{figuras/questionario/resultados-hq-2.png}
      \legend{Fonte: Autores, 2022.}
  \end{center}
\end{figure}

A Figura \ref{fig:hq_respostas_2} representa as respostas acerca do processo de arrastar os requisitos funcionais para dentro dos não funcionais, considerando a House of Quality, trazendo uma maior visão sobre os aspectos de usabilidade da ferramenta \textit{iFlow}. Conclui-se que \textbf{85\%} dos usuários afirmaram que este processo simplifica a criação da \textit{House of Quality}, acarretando um \textit{feedback} positivo.

Por fim, foi perguntado aos usuários o que poderia ser melhorado neste processo da \textit{House of Quality} e, fazendo um apanhado geral das respostas, tem-se que:

\begin{enumerate}
    \item Melhorar o contraste das cores e dar um destaque maior para os requisitos funcionais, além de diferenciar os componentes em tela;
    \item Deixar mais explícito que os requisitos podem ser arrastados e indicar quais elementos são clicáveis, e
    \item Adicionar um \textit{tooltip} para exibir os detalhes sobre cada requisito funcional.
\end{enumerate}

%%%%%%%%%%%%%%%%%%%%%%%%% NFR $$$$$$$$$$$$$$$$$$$$$$$$$$$$$$$$$$$$$$$$$$$$

\subsubsection{\textit{NFR}}
A seção do \textit{NFR}, no questionário, dividiu-se da seguinte forma:

\begin{enumerate}
    \item Uma breve seção explicando como o NFR se aplica no \textit{iFlow};
    \item Uma pequena descrição e um \textit{hyperlink} do teste de usabilidade que foi realizado pelo usuário, e
    \item As perguntas referentes ao teste de usabilidade.
\end{enumerate}

A Figura \ref{fig:nfr_secao} retrata a segunda seção do questionário, onde é retratado o NFR.

\begin{figure}[]
    \begin{center}
        \caption{{Seção do NFR no Questionário desenvolvido}}
        \label{fig:nfr_secao}
        \includegraphics[scale=0.5]{figuras/questionario/nfr-section-1.png}
        \legend{Fonte: Autores, 2022.}
    \end{center}
\end{figure}

Dessa forma, após a realização do teste por parte de cada usuário, foram indagadas as seguintes perguntas:

\begin{itemize}
    \item Quão fácil você achou de realizar a tarefa?
    \item É mais fácil determinar o NFR a partir de um modelo guiado?
    \item Você julga que essas telas são intuitivas?
    \item O que você acredita que poderia ser melhorado nesse processo?
\end{itemize}

\begin{figure}[H]
    \begin{center}
        \caption{{Gráfico retratando as respostas da pergunta 1: Quão fácil você achou de realizar a tarefa?}}
        \label{fig:nfr_respostas_1}
        \includegraphics[scale=0.65]{figuras/questionario/nfr-1.png}
        \legend{Fonte: Autores, 2022.}
    \end{center}
\end{figure}

A Figura \ref{fig:nfr_respostas_1} representa as respostas acerca da facilidade em se realizar a tarefa na etapa do NFR. Conclui-se que grande parte dos usuários conseguiram realizar a tarefa de modo fácil, tendo \textbf{71,8\%} das respostas finais nas notas 4 e 5.

\begin{figure}[H]
    \begin{center}
        \caption{{Gráfico retratando as respostas da pergunta 2: É mais fácil determinar o NFR a partir de um modelo guiado?}}
        \label{fig:nfr_respostas_2}
        \includegraphics[scale=0.65]{figuras/questionario/nfr-2.png}
        \legend{Fonte: Autores, 2022.}
    \end{center}
\end{figure}

A Figura \ref{fig:nfr_respostas_2} representa as respostas acerca da facilidade em se realizar o NFR a partir de um modelo guiado, já que é uma representação bem complexa no método tradicional. Conclui-se que \textbf{84,6\%} dos usuários afirmaram ser de grande utilidade e de grande facilidade realizar o NFR orientando-se por um modelo guiado, trazendo um \textit{feedback} positivo em relação ao método tradicional. Vale ressaltar que essa é uma versão minimizada do NFR, já que o \textit{iFlow} usa apenas três aspectos qualitativos aprofundados em apenas dois níveis.

\begin{figure}[]
    \begin{center}
        \caption{{Gráfico retratando as respostas da pergunta 3: Você julga que essas telas são intuitivas?}}
        \label{fig:nfr_respostas_3}
        \includegraphics[scale=0.65]{figuras/questionario/nfr-3.png}
        \legend{Fonte: Autores, 2022.}
    \end{center}
\end{figure}

A Figura \ref{fig:nfr_respostas_3} demonstra se as telas referentes ao NFR são intuitivas. Conclui-se que \textbf{69,2\%} dos usuários afirmaram que as telas são de fácil entendimento. Entretanto, \textbf{30,8\%} dos usuários afirmaram que o nível de entendimento das telas está numa categoria de regular para ruim, o que mostra que essas telas precisam ser melhoradas para uma versão futura.

Por fim, foi perguntado aos usuários o que poderia ser melhorado neste processo do NFR e, fazendo um conglomerado das respostas, tem-se que:

\begin{enumerate}
    \item Deixar mais claro sobre como relacionar os requisitos não funcionais a cada aspecto;
    \item Acrescentar destaque aos aspectos que estão sendo considerados;
    \item Utilizar mais cores para diferenciar o que cada componente faz, e
    \item Deixar mais claro quais são os botões clicáveis.
\end{enumerate}


%%%%%%%%%%%%%%%%%%%%%%%%%%% BACKLOG %%%%%%%%%%%%%%%%%%%%%%%%%%%
\subsubsection{\textit{Backlog} do Produto}
A seção do \textit{Backlog} do Produto, no questionário, dividiu-se da seguinte forma:

\begin{enumerate}
    \item Uma breve seção explicando como o \textit{Backlog} do Produto se aplica no \textit{iFlow};
    \item Uma pequena descrição e um \textit{hyperlink} do teste de usabilidade que foi realizado pelo usuário, e
    \item As perguntas referentes ao teste de usabilidade.
\end{enumerate}

A Figura \ref{fig:backlog_secao} ilustra a terceira seção do questionário, onde é retratado o \textit{Backlog} do Produto.

\begin{figure}[H]
    \begin{center}
        \caption{{Seção do Backlog no Questionário desenvolvido}}
        \label{fig:backlog_secao}
        \includegraphics[scale=0.5]{figuras/questionario/backlog-section-1.png}
        \legend{Fonte: Autores, 2022.}
    \end{center}
\end{figure}

Dessa forma, após a realização do teste por parte do usuário, foram indagadas as seguintes perguntas:

\begin{itemize}
    \item Quão fácil você achou de realizar a tarefa?
    \item Quão importante você julga que foi arrastar os requisitos em tela?
    \item O que você acredita que poderia ser melhorado nesse processo?
\end{itemize}

\begin{figure}[H]
    \begin{center}
        \caption{{Gráfico retratando as respostas da pergunta 1: Quão fácil você achou de realizar a tarefa?}}
        \label{fig:backlog_respostas_1}
        \includegraphics[scale=0.65]{figuras/questionario/backlog-1.png}
        \legend{Fonte: Autores, 2022.}
    \end{center}
\end{figure}

A Figura \ref{fig:backlog_respostas_1} representa as respostas acerca da facilidade em se realizar a tarefa na etapa do \textit{Backlog} do Produto. Conclui-se que grande parte dos usuários conseguiram realizar a tarefa de modo fácil, tendo \textbf{72,5\%} das respostas finais nas notas 4 e 5. Entretanto, \textbf{15\%} afirmaram que a tarefa teve uma dificuldade regular.

\begin{figure}[H]
    \begin{center}
        \caption{{Gráfico retratando as respostas da pergunta 2: Quão relevante você julga que foi arrastar os requisitos em tela?}}
        \label{fig:backlog_respostas_2}
        \includegraphics[scale=0.65]{figuras/questionario/backlog-2.png}
        \legend{Fonte: Autores, 2022.}
    \end{center}
\end{figure}

A Figura \ref{fig:backlog_respostas_2} demonstra a relevância na funcionalidade de arrastar os requisitos em tela, acarretando em uma maior visão sobre os aspectos de usabilidade da ferramenta \textit{iFlow}. Conclui-se que \textbf{82,5\%} dos usuários tiveram um olhar otimista e positivo em relação a essa funcionalidade.

Por fim, foi perguntado aos usuários o que poderia ser melhorado neste processo do \textit{Backlog} do Produto e, fazendo um apanhado geral das respostas, tem-se que:

\begin{enumerate}
    \item Diferenciar os componentes em tela, pois são muitos e isso pode confundir o usuário;
    \item Usar cores para destacar o que já foi realizado, pois isso atrapalha o usuário, e
    \item Mais informações a respeito do que se deve fazer, por exemplo, indicando que os requisitos podem ser arrastados.
\end{enumerate}

\subsection{Teste de Usabilidade}
Nesta fase, também, foi realizada a coleta de dados, visando validar a usabilidade da ferramenta \textit{iFlow} no processo de Engenharia de Requisitos. O teste foi realizado na plataforma \href{https://usabilityhub.com/}{\textit{Usability Hub}}, criando três testes para as telas mais relevantes do \textit{iFlow} (\textit{House of Quality}, \textit{NFR} e \textit{Backlog} do Produto).

Dessa forma, para se ter uma melhor interpretação dos dados, foram utilizados \textit{Heatmaps} para se ter uma melhor visualização, ou seja, pistas visuais óbvias sobre como o fenômeno está agrupado. O Heatmap, nesse contexto, tem o objetivo de mostrar quais as áreas que foram mais clicadas, colocando uma variação de cor, do mais quente (maior quantidade de cliques), para o mais frio (menor quantidade de cliques).

Os participantes são atuantes na área de \textit{software} e requisitos. Esse público possui experiência prévia dos conceitos inerentes ao escopo deste trabalho, podendo validar a usabilidade da ferramenta \textit{iFlow} com mais propriedade.

\subsubsection{\textit{House of Quality}}

Este cenário de usabilidade consistiu em uma simulação do usuário tentando fazer o relacionamento de um requisito funcional com um não funcional na \textit{interface} do \textit{House of Quality} da ferramenta \textit{iFlow}. As Figuras \ref{fig:hoq_hm_1} e \ref{fig:hoq_hm_2} revelam quais foram os passos do usuário, de forma que se pode observar os pontos mais clicados e menos clicados durante este teste.

\begin{figure}[]
  \begin{center}
      \caption{{\textit{Heatmap} da Tela das Etapas da Engenharia de Requisitos}}
      \label{fig:hoq_hm_1}
      \includegraphics[scale=0.45]{figuras/UsabilityHub/hoq/1.png}
      \includegraphics[scale=0.45]{figuras/UsabilityHub/hoq/2.png}
      \legend{Fonte: Autores, 2022.}
  \end{center}
\end{figure}

\begin{figure}[]
  \begin{center}
      \caption{{\textit{Heatmap} da Tela de Criação do \textit{House of Quality}}}
      \label{fig:hoq_hm_2}
      \includegraphics[scale=0.45]{figuras/UsabilityHub/hoq/3.png}
      \includegraphics[scale=0.45]{figuras/UsabilityHub/hoq/4.png}
      \legend{Fonte: Autores, 2022.}
  \end{center}
\end{figure}

Com base nos dados expostos, podemos tirar as seguintes conclusões:
\begin{enumerate}
    \item O primeiro ponto a ser observado nesse teste se dá pelo fato de os usuários não conhecerem a plataforma e, por isso, foram clicando de forma aleatória até se familiarizar com a ferramenta, e
    \item Ainda que de forma um tanto quanto rústica, por não ser possível ainda realizar o movimento de arrastar nas navegações do protótipo, pode-se observar que os usuários conseguiram chegar ao requisito funcional determinado pelo cenário (Figura \ref{fig:hq_secao}).
\end{enumerate}

\subsubsection{NFR}

O intuito deste cenário de usabilidade foi o de simular o usuário seguindo os passos determinados para gerar o \textit{NFR} na ferramenta \textit{iFlow}. As Figuras \ref{fig:nfr_hm_1}, \ref{fig:nfr_hm_2}, \ref{fig:nfr_hm_3}, \ref{fig:nfr_hm_4} e \ref{fig:nfr_hm_5} revelam quais foram os passos do usuário, de forma que se pode observar os pontos mais clicados e menos clicados durante este teste.

\begin{figure}[]
  \begin{center}
      \caption{{\textit{Heatmap} da Tela de Criação dos Artefatos da Etapa de Modelagem}}
      \label{fig:nfr_hm_1}
      \includegraphics[scale=0.45]{figuras/UsabilityHub/nfr/1.png}
      \includegraphics[scale=0.45]{figuras/UsabilityHub/nfr/2.png}
      \legend{Fonte: Autores, 2022.}
  \end{center}
\end{figure}

\begin{figure}[]
  \begin{center}
      \caption{{\textit{Heatmap} da Tela de Criação do NFR no seu Primeiro Nível}}
      \label{fig:nfr_hm_2}
      \includegraphics[scale=0.43]{figuras/UsabilityHub/nfr/3.png}
      \includegraphics[scale=0.43]{figuras/UsabilityHub/nfr/4.png}
      \legend{Fonte: Autores, 2022.}
  \end{center}
\end{figure}

\begin{figure}[]
  \begin{center}
      \caption{{\textit{Heatmap} da Tela de Criação do NFR no seu Segundo Nível}}
      \label{fig:nfr_hm_3}
      \includegraphics[scale=0.45]{figuras/UsabilityHub/nfr/5.png}
      \includegraphics[scale=0.45]{figuras/UsabilityHub/nfr/6.png}
      \legend{Fonte: Autores, 2022.}
  \end{center}
\end{figure}

\begin{figure}[]
  \begin{center}
      \caption{{\textit{Heatmap} da da Tela de Criação do NFR no seu Terceiro Nível}}
      \label{fig:nfr_hm_4}
      \includegraphics[scale=0.45]{figuras/UsabilityHub/nfr/7.png}
      \includegraphics[scale=0.45]{figuras/UsabilityHub/nfr/8.png}
      \legend{Fonte: Autores, 2022.}
  \end{center}
\end{figure}

\begin{figure}[]
  \begin{center}
      \caption{{\textit{Heatmap} da Tela de Criação do NFR com Todos os Níveis Preenchidos}}
      \label{fig:nfr_hm_5}
      \includegraphics[scale=0.45]{figuras/UsabilityHub/nfr/9.png}
      \includegraphics[scale=0.45]{figuras/UsabilityHub/nfr/10.png}
      \legend{Fonte: Autores, 2022.}
  \end{center}
\end{figure}


Com base nos dados expostos, podemos tirar as seguintes conclusões:

\begin{enumerate}
    \item No escopo da seleção/criação do primeiro requisito não-funcional, pode-se notar certa dificuldade dos usuários em saber aonde clicar. Isso pode ter sido dado pela diferenciação fraca entre componentes clicáveis e não clicáveis, e
    \item Realizada a primeira definição do requisito não-funcional, o usuário conseguiu seguir com a criação do NFR como o esperado.
\end{enumerate}


\subsubsection{\textit{Backlog} do Produto}

A finalidade deste cenário de usabilidade foi o de reproduzir a criação de um \textit{Backlog} de produto na ferramenta \textit{iFlow}. As Figuras \ref{fig:backlog_hm_1}, \ref{fig:backlog_hm_2}, \ref{fig:backlog_hm_3}, \ref{fig:backlog_hm_4} e \ref{fig:backlog_hm_5} demonstram quais foram os passos do usuário, de forma que se pode observar os pontos mais clicados e menos clicados durante este teste.

\begin{figure}[]
  \begin{center}
      \caption{{\textit{Heatmap} da Tela de Criação dos Artefatos da Etapa de Modelagem}}
      \label{fig:backlog_hm_1}
      \includegraphics[scale=0.45]{figuras/UsabilityHub/backlog/1.png}
      \includegraphics[scale=0.45]{figuras/UsabilityHub/backlog/2.png}
      \legend{Fonte: Autores, 2022.}
  \end{center}
\end{figure}

\begin{figure}[]
  \begin{center}
      \caption{{\textit{Heatmap} da Tela de Criação do \textit{Backlog}}}
      \label{fig:backlog_hm_2}
      \includegraphics[scale=0.45]{figuras/UsabilityHub/backlog/3.png}
      \includegraphics[scale=0.45]{figuras/UsabilityHub/backlog/4.png}
      \legend{Fonte: Autores, 2022.}
  \end{center}
\end{figure}

\begin{figure}[]
  \begin{center}
      \caption{{\textit{Heatmap} da Tela de Criação do \textit{Backlog} com o Requisito sendo Arrastado}}
      \label{fig:backlog_hm_3}
      \includegraphics[scale=0.45]{figuras/UsabilityHub/backlog/5.png}
    \includegraphics[scale=0.45]{figuras/UsabilityHub/backlog/6.png}
    \legend{Fonte: Autores, 2022.}
\end{center}
\end{figure}

\begin{figure}[H]
  \begin{center}
      \caption{{\textit{Heatmap} da Tela de Criação de um novo Requisito Funcional ou Requisito não Funcional}}
      \label{fig:backlog_hm_4}
      \includegraphics[scale=0.45]{figuras/UsabilityHub/backlog/7.png}
    \includegraphics[scale=0.45]{figuras/UsabilityHub/backlog/8.png}
    \legend{Fonte: Autores, 2022.}
\end{center}
\end{figure}

\begin{figure}[]
  \begin{center}
      \caption{{\textit{Heatmap} da Tela do \textit{Backlog} preenchido com Épicos, \textit{Features} e História de Usuário}}
      \label{fig:backlog_hm_5}
      \includegraphics[scale=0.45]{figuras/UsabilityHub/backlog/9.png}
    \includegraphics[scale=0.45]{figuras/UsabilityHub/backlog/10.png}
    \legend{Fonte: Autores, 2022.}
\end{center}
\end{figure}

Com base nos dados expostos, podemos tirar as seguintes conclusões:

\begin{enumerate}
    \item Este processo pode ser considerado um dos mais trabalhosos dentre os passos necessários da Engenharia de Requisitos, e a interface proposta mostrou-se com uma grande taxa de sucesso, uma vez que ainda que tenham alguns cliques fora do componente esperado, os cliques estão mais concentrados no componente esperado;
    \item Na Figura \ref{fig:backlog_hm_4}, pode-se observar que a confusão de cliques ocorreu pelos usuários quererem preencher as informações solicitadas no formulário, o que se mostra positivo, pois pode evidenciar uma interface simples e direta, e
    \item A confusão de cliques da Figura \ref{fig:backlog_hm_5} pode ter se dado pela falta de evidência de que o requisito não estava mais disponível para ser usado.
\end{enumerate}

\section{Elaboração do Plano de Ação}
\label{sec:plano_de_acao}

A partir dos \textit{feedbacks} coletados com o questionário e com o teste de usabilidade, e avaliando as análises descritas nas seções anteriores, foi possível levantar alguns pontos para melhorar o \textit{iFlow}, resultando em um Plano de Ação. Dentre os pontos de melhorias, compreendidos no Plano de Ação, destacam-se:

\begin{enumerate}
    \item Melhoria de contraste no uso das cores presentes nas telas e nos componentes;
    \item Adição de um componente, na tela de \textit{House of Quality}, para poder visualizar as informações dos requisitos funcionais;
    \item Adicionar mais cores para poder destacar e evidenciar cada componente e facilitar o desenvolvimento das tarefas correspondentes;
    \item Deixar mais evidente quais componentes são clicáveis;
    \item Melhorar, na tela do \textit{NFR}, como se relacionam os aspectos qualitativos propostos com os requisitos que estão sendo destrinchados;
    \item Adição de mensagens intuitivas que evidenciem que os componentes em tela são arrastáveis, e guiar para aonde devem ser arrastados, e
    \item Adicionar evidências para quando um componente já não estiver mais disponível para ser arrastado, na tela de \textit{Backlog}.
\end{enumerate}

\section{Divulgação de Resultados}
\label{sec:divulgacao_resultados}
Diante do abrangente escopo de atuação que foi definido para a ferramenta \textit{iFlow}, sendo, de fato, automatizada boa parte do processo da Engenharia de Requisitos com um enfoque em gerar um Produto Mínimo Viável, em sua versão preliminar, a presente Pequisa-Ação teve o intuito de levantar pontos de melhoria da ferramenta e definir um conjunto de próximos passos para solucioná-los. As implementações dessas melhorias compreendem excelentes oportunidades para trabalhos futuros, permitindo refinar ainda mais a ferramenta \textit{iFlow}, e conferindo mais fidedignidade e contribuições às atividades da Engenharia de Requisitos.


\section{Considerações Finais}
\label{sec:consideracoes_finais_analise}
Neste capítulo, foram apresentados os resultados obtidos ao longo das fases de Pesquisa-Ação. A fase de Coleta de Dados identifica o problema e o contexto em que este trabalho está inserido, visando coletar dados para as fases posteriores. Em seguida, a fase de Análise e Interpretação de Dados aborda o estudo feito para a Coleta de Dados, provida consultando os usuários participantes da pesquisa. Logo em seguida, tem-se a Elaboração de um Plano de Ação que, a partir dos dados coletados, foram planejadas ações futuras para atender às sugestões dos usuários participantes do questionário e do teste de usabilidade aplicados. Por fim, tem-se a divulgação dos resultados provindos do Plano de Ação, em que estes ficam como oportunidade para trabalhos futuros.